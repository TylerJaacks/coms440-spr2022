
\documentclass{article}
\usepackage{fullpage}
\usepackage{url}
\usepackage{color}
\usepackage{listings}
\usepackage{longtable}

\renewcommand{\topfraction}{0.99}

\definecolor{dkgreen}{rgb}{0,0.6,0}
\definecolor{muave}{rgb}{0.58,0,0.82}
\definecolor{dkred}{rgb}{0.6,0,0}
\definecolor{dkblue}{rgb}{0,0,0.7}

\lstset{frame=none,
  language=C,
  % aboveskip=3mm,
  % belowskip=3mm,
  xleftmargin=2em,
  % xrightmargin=2em,
  showstringspaces=false,
  columns=flexible,
  basicstyle={\ttfamily},
  numbers=left,
  stepnumber=1,
  keywordstyle=\color{dkblue},
  directivestyle=\color{dkred},
  commentstyle=\color{dkgreen},
  stringstyle=\color{muave},
  breaklines=true,
  breakatwhitespace=true,
  tabsize=2
}

\lstdefinestyle{Output}{
  % aboveskip=3mm,
  % belowskip=3mm,
  xleftmargin=2em,
  % xrightmargin=2em,
  showstringspaces=false,
  columns=flexible,
  basicstyle={\ttfamily},
  numbers=none,
  keywordstyle=,
  directivestyle=,
  commentstyle=,
  stringstyle=,
  tabsize=8
}

\newcommand{\cpp}{C{\tt ++}}

\newcommand{\gradeline}{ \cline{1-2} \cline{4-4} ~\\[-1.5ex] }

\newenvironment{gradetable}{\begin{longtable}{@{}rrcp{5in}} \multicolumn{2}{l}{\bf Points} & & {\bf Description}\\ \gradeline}{\end{longtable}}

\newcommand{\mainitem}[2]{\pagebreak[2] {\bf #1} &&& {\bf #2}}
\newcommand{\mainpara}[1]{~ &&& {#1} }
\newcommand{\inneritem}[2]{~ & #1 && #2}
\newcommand{\innerpara}[1]{~ & ~ && #1}

\newcommand{\iseasier}{is$^\dagger$ }
\newcommand{\isextra}{is$^\ddagger$ }

\newcommand{\thispart}{3}
\newcommand{\typecheck}{4}
\newcommand{\codegen}{5}

\title{COM S 440/540 Project part \thispart}
\author{C parser}
\date{}

\begin{document}

\maketitle


%======================================================================
\section{Requirements for part \thispart}
%======================================================================

When executed with a mode of \thispart,
your compiler should read the specified input file
and check that the file has correct C syntax
(for our subset of C, discussed in Section~\ref{SEC:language}).
If the input file is not syntactically correct,
then display an appropriate error message.
Error messages should be written to standard error,
and have the format
\begin{quote}
  \begin{tabbing}
		{\tt Parser }\={\tt error in file }\emph{filename}
		{\tt line }\emph{line number}
		{\tt near text }\emph{lexeme}
	\\
		\> \emph{Description}
  \end{tabbing}
\end{quote}
where the line number and offending text should
be where the error occurs (or was detected).
The description will typically indicate
what the parser expected to appear next in input.
After the first syntax error,
your parser may either make a ``best effort''
attempt to continue processing the input file, or exit.
When tested on input files with syntax errors,
your compiler will be considered correct if it catches the first syntax error.
If the input file is syntactically correct,
then simply write
\begin{quote}
  \verb|File| \emph{filename} \verb|is syntactically correct.|
\end{quote}
to the output stream.

This part of the project checks {\bf only} the input file {\bf syntax}.
You {\bf should not} check for type consistency,
for existence of called functions,
for existence of used variables,
for duplication of function definitions,
  or for duplication of variable declarations.
All of these will be required instead for part~\typecheck{} of the project.


%======================================================================
\section{C language (subset) definition} \label{SEC:language}
%======================================================================

For a minimalist implementation,
the input language is defined in Section~\ref{SEC:minimal}.
Extra credit options,
  potentially requiring grammar modifications,
are discussed in Sections~\ref{SEC:firstextra}
through~\ref{SEC:lastextra}.
Rules that are more restrictive than the C standard
(for purposes of making the parser or compiler simpler)
are marked with $\dagger$ or $\ddagger$,
with $\ddagger$ indicating that the rule will need to be modified
to complete an extra credit option.

\subsection{Minimalist rules}
\label{SEC:minimal}

From the rules below,
you should be able to define an appropriate grammar
for our subset of the C programming language.
The rules are numbered for later reference.
%
%
\begin{enumerate}
\item\label{Cprog}
  A C program \isextra a sequence of zero or more
  (global) variable declarations, function prototypes, and function definitions,
  appearing in any order.

\item\label{vardec}
  A \emph{variable declaration} \isextra a type name,
  followed by a comma-separated list of one or more identifiers,
  each identifier optionally followed by a left bracket,
  an integer constant, and a right bracket.
  The list is terminated with a semicolon.
  Note that this restricts arrays to a single dimension.

\item\label{typename}
  A \emph{type name} \isextra one of the simple types:
    {\tt void}, {\tt char}, {\tt int}, {\tt float}.

\item
  A \emph{function prototype} is a function declaration
  followed by a semicolon.

\item
  A \emph{function declaration} is a type name
  (the return type of the function),
  followed by an identifier (the name of the function),
  a left parenthesis,
    an optional comma-separated list of formal parameters,
  and a right parenthesis.

\item
  A \emph{formal parameter} \iseasier a type name,
  followed by an identifier,
  and optionally followed by a left and right bracket.

\item \label{funcdef}
  A \emph{function definition} \isextra a function declaration
  followed by a left brace,
  a sequence of zero or more
    variable declarations or statements,
  and a right brace.

\item
  A \emph{statement block} is a left brace,
  a sequence of zero or more statements,
  and a right brace.

\item
  A \emph{statement} \iseasier one of the following.

  \begin{itemize}
    \item Nothing, followed by a semicolon.
    \item An expression followed by a semicolon.

    \item Keywords {\bf break} or {\bf continue} followed by a semicolon.
    Note: you do {\bf not} need to check that these statements
      are within a loop (yet).

    \item Keyword {\bf return}, followed by an optional expression,
    and a semicolon.

    \item Keyword {\bf if}, followed by a left parenthesis,
    an expression, and a right parenthesis,
    followed by either a statement block or a single statement.
    Then, optionally, the following:
    keyword {\bf else}, followed by either a statement block,
    or a single statement.

    \item Keyword {\bf for}, followed by a left parenthesis,
      an optional expression, a semicolon, an optional expression,
      a semicolon, an optional expression, a right parenthesis,
      and then either a statement block,
      or a single statement.

    \item Keyword {\bf while}, followed by a left parenthesis,
      an expression, and a right parenthesis,
      and then either a statement block, or a single statement.

    \item Keyword {\bf do}, followed by either a statement block
    or a single statement,
    followed by keyword {\bf while}, a left parenthesis, an expression,
    a right parenthesis, and a semicolon.

  \end{itemize}

\item
  An \emph{expression} \iseasier one of the following.
  \begin{itemize}
    \item A literal (constant) value.
    \item An identifier, left parenthesis,
    a comma-separated list of zero or more expressions,
    and a right parenthesis.

    \item An l-value.

    \item An l-value, an \emph{assignment operator}, and an expression.

    \item An l-value, preceeded by or followed by the
    increment or decrement operator.

    \item A \emph{unary operator}, and an expression.

    \item An expression, a \emph{binary operator}, and an expression.

    \item An expression, a question mark, an expression, a colon,
    and an expression.

    \item A left parenthesis, a type name, a right parenthesis, and
    an expression.
    \item A left parenthesis, an expression, and a right parenthesis.
  \end{itemize}

\item \label{lvalue}
  An \emph{l-value} \isextra an identifier,
  optionally followed by a left bracket,
  an expression, and a right bracket.
  Note that this restricts array access to a single dimension.

\item
  Unary operators (for any expression) are: \verb|-, !, ~|

\item
  Binary operators are: \verb$==, !=, >, >=, <, <=, +, -, *, /, %, |, &, ||, &&$

\item
  Assignment operators are: \verb$=, +=, -=, *=, /=$
\end{enumerate}
%
%
Operator precedence and rules for associativity
are shown in Table~\ref{TAB:precedence}.
%
\begin{table}[t]
\begin{center}
\begin{tabular}{@{}l|c|c}
\hline
  \multicolumn{1}{c|}{\sc Operators}
&
  {\sc Associativity}
&
  {\sc Precedence}
\\ \hline
  \verb|()| \quad \verb|[]| \quad .
  &
  left to right
  &
  (highest)
\\
  \verb|!| \quad \verb|~| \quad \verb|-| (unary)
  \quad \verb|--| \quad \verb|++|
  % \quad \verb|&| (unary)
  \quad \verb|(|\emph{type}\verb|)|
  &
  right to left
\\
  \verb|*| \quad \verb|/| \quad \verb|%|
  &
  left to right
\\
  \verb|+| \quad \verb|-|
  &
  left to right
\\
  \verb|<| \quad \verb|<=| \quad \verb|>| \quad \verb|>=|
  &
  left to right
\\
  \verb|==| \quad \verb|!=|
  &
  left to right
\\
  \verb|&|
  &
  left to right
\\
  \verb$|$
  &
  left to right
\\
  \verb|&&|
  &
  left to right
\\
  \verb$||$
  &
  left to right
\\
  \verb$?:$
  &
  right to left
\\
  \verb|=| \quad \verb|+=| \quad \verb|-=| \quad \verb|*=| \quad \verb|/=|
  &
  right to left
\\
  \verb|,|
  &
  left to right
  &
  (lowest)
\\ \hline
\end{tabular}
\end{center}
\caption{Precedence and associativity of operators}
\label{TAB:precedence}
\end{table}


\subsection{Extra credit: variable initialization}
\label{SEC:firstextra}
\label{SEC:varinit}

Relax rule~\ref{vardec} to allow global and local variables to be
declared and initialized at the same time.
For example, this would allow variable declarations of the form
\begin{lstlisting}[numbers=none]
  int a, b=3, c, d=b+1;
\end{lstlisting}
for both global and local variables.
You may do this for simple types only;
do not worry about initializing arrays or struct variables.

If you choose to implement this feature,
you may have the option of earning
additional extra credit for code generation
in part~\codegen.

\subsection{Extra credit: constants}
\label{SEC:constants}

Relax rule~\ref{typename} to allow type names to be modified
with the keyword {\tt const}, either before or after the type name.
For example,
this would enable the declaration of function parameters
with type \lstinline{const int} or \lstinline{int const};
note that these are equivalent types.

If you choose to implement this feature,
you will have the option of earning
additional extra credit for type checking
in part~\typecheck.


\subsection{Extra credit: user-defined structs}
\label{SEC:userstructs}

Modify rule~\ref{typename} so that keyword {\tt struct},
followed by an identifier,
is also a valid type name.
If you implement the {\tt const} keyword (c.f. Section~\ref{SEC:constants}),
make sure it may be applied to {\tt struct} types
and members.

Modify rules~\ref{Cprog} and \ref{funcdef} to allow user-defined
{\tt struct} types.
For a C program, global user-defined types may appear in any order.
While any type and variable declarations that are local to a function
may be intermingled,
you may assume (and define your grammar accordingly)
that these must appear before any statements of the function.
A \emph{user-defined type declaration} \iseasier the keyword {\tt struct},
followed by an identifier, a left brace,
  zero or more variable declarations (without initializations),
a right brace, and a semicolon.

Note that you must allow (1) arrays of structs, (2) array variables
inside structs, and (3) arbitrary nesting of structs.
%
If you choose to implement this feature,
you will have the option of earning
additional extra credit for type checking
in part \typecheck.

\subsection{Extra credit: struct member selection}
\label{SEC:lastextra}
\label{SEC:members}

Modify rule~\ref{lvalue} to allow member selection using the
{\tt .} operator.
For example,
\lstinline{point.x}
should be a valid l-value.
Remember that you must allow (1) arrays of structs,
(2) array variables inside structs, and (3) arbitrary nesting of structs.
For example,
\begin{lstlisting}[numbers=none]
  window[3].upperleft.x = mouse.x.stack[mouse.x.top];
\end{lstlisting}
should be syntactically correct.
The \emph{type checking} of that expression falls under part \typecheck,
where you will have the option of earning additional extra credit.



%======================================================================
\section{Examples: minimal implementation} \label{SEC:exmin}
%======================================================================

\subsection{Input: {\tt test1.c}}

\lstinputlisting{test1.c}

\subsection{Output for {\tt mycc -\thispart{} test1.c}}

\lstinputlisting[style=Output]{test1.c.output3}

\subsection{Errors for {\tt mycc -\thispart{} p2test.c}}

The error stream should be empty.

\subsection{Input: {\tt test2.c}}

\lstinputlisting{test2.c}

\subsection{Errors for {\tt mycc -\thispart{} test2.c}}

\lstinputlisting[style=Output]{test2.c.error3}

%======================================================================
\section{Examples: extra credit} \label{SEC:exmax}
%======================================================================

\subsection{Input: {\tt test3.c}}

\lstinputlisting{test3.c}

\subsection{Errors for basic implementation, on {\tt test3.c}}

\lstinputlisting[style=Output]{test3.c.error3}

\subsection{Output for extra credit implementation, on {\tt test3.c}}

\lstinputlisting[style=Output]{test3.c.extra.output3}

%======================================================================
\section{Grading}
%======================================================================

\noindent
\begin{gradetable}
  \mainitem{17}{Documentation}
  \\[1mm]
  \inneritem{5}{\tt README.txt}
  \\[1mm]
  \innerpara{%
    How to build the compiler and documentation.
    Updated to show which part \thispart{} features are implemented.
  }
  \\[1mm]
  \inneritem{12}{\tt developers.pdf}
  \\[1mm]
  \innerpara{%
    New section for part \thispart{}, that explains
    the purpose of each source file,
    the main data structures used,
    and gives a high-level overview of how the various
    features are implemented.
  }
  \\[4mm]

  \mainitem{8}{Ease of building}
  \\[1mm]
  \mainpara{%
    How easy was it for the graders to build your compiler and
    documentation from the {\tt README} file.
		You are encouraged to use {\tt Makefile}s under the
		{\tt Source/} and {\tt Documentation/}
		directories so that running ``{\tt make}''
		will build everything and running ``{\tt make clean}''
		will remove all generated files.
  }
  \\[4mm]

  \mainitem{5}{{\tt -o} switch works}
  \\[4mm]
  \mainitem{5}{Still works in mode 0}
  \\[4mm]
  \mainitem{5}{Still works in mode 1}
  \\[4mm]

  \mainitem{60}{Basic Parser}
  \\[2mm]
    \inneritem{5}{Global variables}
  \\
    \inneritem{5}{Function prototypes / parameter lists}
  \\
    \inneritem{5}{Function local variables and body}
  \\[2mm]
    \inneritem{10}{For, while, do loops}
  \\
    \inneritem{5}{if then else}
  \\
    \inneritem{5}{break / continue / return / expression stmts}
  \\[2mm]
    \inneritem{10}{Expressions with unary/binary/ternary operators}
  \\
    \inneritem{5}{Assignment operators; increment and decrement}
  \\
    \inneritem{5}{Identifiers and arrays}
  \\
    \inneritem{5}{Function calls and parameters}
  \\[4mm]

  \mainitem{5}{Extra credit: Variable initializations}
  \\[1mm]
    \innerpara{See Section~\ref{SEC:varinit}}
  \\[4mm]

  \mainitem{5}{Extra credit: Constants}
  \\[1mm]
    \innerpara{See Section~\ref{SEC:constants}}
  \\[4mm]

  \mainitem{10}{Extra credit: User-defined structs}
  \\[1mm]
    \innerpara{See Section~\ref{SEC:userstructs}}
  \\[4mm]

  \mainitem{5}{Extra credit: Struct member selection}
  \\[1mm]
    \innerpara{See Section~\ref{SEC:members}}
  \\[4mm]

  \gradeline
  \mainitem{100}{Total for students in 440 (max points is 120)}
  \\
  \mainitem{115}{Total for students in 540}
\end{gradetable}


\section{Submission}

\begin{table}[h]
\centering

  \begin{tabular}{lcr}
    {\bf Part} & ~~~~ & {\bf Penalty applied} \\ \cline{1-1}\cline{3-3}
    Part 0 && 20\% off \\
    Part 1 && 10\% off
  \end{tabular}

\caption{Penalty applied when re-grading}
\label{TAB:penalties}
\end{table}

Be sure to commit your source code and documentation to your
git repository, and to upload (push) those commits to the server
so that we may grade them.
In Canvas,
indicate which parts you would like us to re-grade for reduced credit
(see Table~\ref{TAB:penalties} for penalty information).
Otherwise, we will grade only part \thispart.


\end{document}

