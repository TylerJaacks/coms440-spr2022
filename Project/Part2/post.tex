
\documentclass{article}
\usepackage{fullpage}
\usepackage{url}
\usepackage{color}
\usepackage{listings}
\usepackage{longtable}

\renewcommand{\topfraction}{0.99}

\definecolor{dkgreen}{rgb}{0,0.6,0}
\definecolor{muave}{rgb}{0.58,0,0.82}
\definecolor{dkred}{rgb}{0.6,0,0}
\definecolor{dkblue}{rgb}{0,0,0.7}

\lstset{frame=none,
  language=C,
  % aboveskip=3mm,
  % belowskip=3mm,
  xleftmargin=2em,
  % xrightmargin=2em,
  showstringspaces=false,
  columns=flexible,
  basicstyle={\ttfamily},
  numbers=left,
  stepnumber=1,
  keywordstyle=\color{dkblue},
  directivestyle=\color{dkred},
  commentstyle=\color{dkgreen},
  stringstyle=\color{muave},
  breaklines=true,
  breakatwhitespace=true,
  tabsize=2
}

\lstdefinestyle{Output}{
  % aboveskip=3mm,
  % belowskip=3mm,
  xleftmargin=2em,
  % xrightmargin=2em,
  showstringspaces=false,
  columns=fixed,
  basicstyle={\ttfamily},
  numbers=none,
  keywordstyle=,
  directivestyle=,
  commentstyle=,
  stringstyle=,
  tabsize=8
}

\newcommand{\cpp}{C{\tt ++}}

\newcommand{\gradeline}{ \cline{1-2} \cline{4-4} ~\\[-1.5ex] }

\newenvironment{gradetable}{\begin{longtable}{@{}rrcp{5in}} \multicolumn{2}{l}{\bf Points} & & {\bf Description}\\ \gradeline}{\end{longtable}}

\newcommand{\mainitem}[2]{\pagebreak[2] {\bf #1} &&& {\bf #2}}
\newcommand{\mainpara}[1]{~ &&& {#1} }
\newcommand{\inneritem}[2]{~ & #1 && #2}
\newcommand{\innerpara}[1]{~ & ~ && #1}


\title{COM S 440/540 Project part 2, entirely extra credit}
\author{C Preprocessor}
\date{}

\begin{document}

\maketitle

%======================================================================
\section{Overview}
%======================================================================

When executed with a mode of 2,
your compiler should read the specified input file
and divide it into tokens that are written to the output stream.
The format is similar to (and backward compatible with) part 1:
each token should generate an output line of the form
\begin{quote}
  \emph{location} \verb|Token| \emph{number} \verb|Text| \emph{lexeme}
\end{quote}
where \emph{location} may be fancier than part~1 (see below).
Extra credit may be earned by implementing certain preprocessor directives.

All implementation must be C, C++, or Java.
Instructor permission is required to use anything beyond
the standard libraries for these languages.
Submissions will be graded on {\tt pyrite.cs.iastate.edu},
and therefore must build and run correctly there.

%======================================================================
\section{Preprocessor directives}
%======================================================================

All preprocessor directives will have the form
\begin{quote}
\begin{verbatim}
  # directive possible other text newline
\end{verbatim}
\end{quote}
i.e., will begin with the \verb|#| character
and end with a newline character.
However, note that comments within a preprocessor directive
must be ignored and treated accordingly:
C-style comments are equivalent to a space,
while C++-style comments are equivalent to a newline.
Note that whitespace (including comments!) may appear
between the \verb|#| and the directive.
Errors should be generated for badly-formed or unrecognized
preprocessor directives.

Conceptually,
the preprocessor sits between the lexer and the parser.
As such, preprocessor directives will be described in terms
of \emph{tokens}.
Students may, at their discretion,
expand the list of tokens recognized by the lexer as needed to
support the preprocessor,
for example by adding the \verb|#| symbol
and special keywords for preprocessor directives
{\bf that are only recognized immediately following} \verb|#|.


%-----------------------------------------------------------------
\subsection{include}
%-----------------------------------------------------------------

The \verb|include| directive has the form
\begin{quote}
  \verb|#| \verb|include| \emph{filename}
\end{quote}
where \emph{filename} is a string literal.
You do not need to support ``system'' includes\footnote{We
  will not test for this, so if you prefer to
  support those as well, you will not lose points.}
that use angled brackets, for example \verb|#include <stdio.h>|.
This directive switches the token input stream to the
indicated file
(with an error generated if the file cannot be opened).
When the token stream from that file is exhausted,
it reverts back to the original file.
Of course, the included file may itself have include directives,
which should be followed recursively.
You may set a reasonable limit to the include depth (say, 256)
with an appropriate error message if this depth is exceeded.

In mode~2, your compiler should write a line to the output stream
of the form
\begin{quote}
  \emph{location} \verb|include expansion|
\end{quote}
when an include directive is reached,
i.e., immediately before switching the token stream.

For extra extra credit,
the preprocessor should check for ``include cycles''
(for example, if file {\tt A} includes file {\tt B} which includes file {\tt C}
which includes file {\tt A})
and print an appropriate error message when this occurs.

%-----------------------------------------------------------------
\subsection{define and undefine}
%-----------------------------------------------------------------

The \verb|define| directive has the form
\begin{quote}
  \verb|#| \verb|define| \emph{macro} \emph{replacement}
\end{quote}
where \emph{macro} is an identifier\footnote{You
  do not need to support macros with parameters.}
and \emph{replacement} is any replacement text, including none.
This directive defines a preprocessor identifier \emph{macro}
with the specified replacement text.
Then,
whenever the input stream encounters an identifier
that exactly matches the \emph{macro},
the token stream switches to the replacement text.
Once the replacement text is exhausted,
the token stream reverts back to its original source.

The \verb|undef| directive has the form
\begin{quote}
  \verb|#| \verb|undef| \emph{macro}
\end{quote}
where \emph{macro} is an identifier.
This directive removes the definition for preprocessor identifier \emph{macro},
if one exists.


In mode~2, your compiler should write a line to the output stream
of the form
\begin{quote}
  \emph{location} \verb|macro expansion|
\end{quote}
when a defined macro identifier is encountered.
For extra extra credit,
the preprocessor should allow \emph{nested macros},
i.e., if the replacement text itself contains
a macro identifier,
then it should be recursively expanded.
You may set a reasonable limit to the macro nesting depth (say, 256)
with an appropriate error message if this depth is exceeded.
For extra extra extra credit,
the preprocessor should detect replacement cycles
and issue an appropriate error message.

%-----------------------------------------------------------------
\subsection{ifdef}
%-----------------------------------------------------------------

The \verb|ifdef| directive has the form
\begin{quote}
  \verb|#| \verb|ifdef| \emph{macro}
\end{quote}
where \emph{macro} is an identifier.
If \emph{macro} is currently defined
(via a \verb|define| directive),
then the token stream remains on;
otherwise, the token stream is switched off.
This lasts until either a matching \verb|else| directive,
in which case the token stream is toggled,
or a matching \verb|endif| directive.
An \verb|ifdef| directive may have at most one \verb|else| directive.
The preprocessor should issue an error if the \verb|ifdef| directive
is not closed by an \verb|endif| directive,
or if an \verb|else| or \verb|endif| directive
does not have a corresponding \verb|ifdef| or \verb|ifndef| directive.

The \verb|ifndef| directive has the form
\begin{quote}
  \verb|#| \verb|ifndef| \emph{macro}
\end{quote}
where \emph{macro} is an identifier.
If \emph{macro} is currently defined
(via a \verb|define| directive),
then the token stream is switched off;
otherwise, the token stream remains on.
Like \verb|ifdef|, the \verb|ifndef| directive
may have at most one \verb|else| directive,
and must be closed by an \verb|endif| directive.

Note that other preprocessor directives,
including \verb|ifdef| and \verb|ifndef|,
may appear within an \verb|ifdef| or \verb|ifndef| directive.
In this case, if the token stream is switched off,
the enclosed preprocessor directive(s) must be
parsed as if the token stream were on,
but not acted upon.


%======================================================================
\section{Formatting}
%======================================================================

\subsection{Compiler output}

The \emph{location} information displayed for each token
and for preprocessor actions
has the following form.
If the input stream is a file,
then \emph{location} should be the text
\begin{quote}
  \verb|File| \emph{filename} \verb|Line| \emph{line number}
\end{quote}
and if the input stream is the replacement text of a macro,
the \emph{location} should be the text
\begin{quote}
  \verb|Macro| \emph{macroname}
\end{quote}
Additionally, the \emph{location} should be indented
by $2d$ \verb|'.'| characters,
where $d \ge 0$ is the recursion depth
caused by sequences of include directives and macro substitutions.

\subsection{Error messages}

Error messages generated by the preprocessor must
be written to standard error,
and have exactly the form
\begin{quote}
	\begin{tabbing}
		{\tt Preprocessor} \= {\tt error in file} \emph{filename}
		{\tt line} \emph{line number}
	\\
		\> \emph{Description}
	\end{tabbing}
\end{quote}
The \emph{Description} should give a helpful description of the error,
and may span more than one line if necessary.
The error stream must be empty if no errors or warnings are generated.


Example inputs, outputs, and errors are given below.
Additionally, students are encouraged to check the output and error messages
given for the example test files, for all parts of the project.
For test files with errors,
code is considered correct if the first error messages given
occur on the same input line and the error messages describe the
same error (as determined by a human).



%======================================================================
\section{Include example}
%======================================================================

\subsection{include1.h}

\lstinputlisting{include1.h}

\subsection{include2.h}

\lstinputlisting{include2.h}

\subsection{include3.h}

\lstinputlisting{include3.h}

\subsection{Output for include1.h in mode 2}

\lstinputlisting[style=Output]{include1.h.output2}


%======================================================================
\section{Macro example}
%======================================================================

\subsection{macros.c}

\lstinputlisting{macros.c}

\subsection{Output for macros.c in mode 2}

\lstinputlisting[style=Output]{macros.c.output2}


%======================================================================
\section{Ifdef example}
%======================================================================

\subsection{ifdefs.c}

\lstinputlisting{ifdefs.c}

\subsection{Output for ifdefs.c in mode 2}

\lstinputlisting[style=Output]{ifdefs.c.output2}

\subsection{ifdef-bad.c}

\lstinputlisting{ifdef-bad.c}

\subsection{Error stream for ifdef-bad.c in mode 2}

\lstinputlisting[style=Output]{ifdef-bad.c.error2}


%======================================================================
\section{Grading}
%======================================================================

\noindent
\begin{gradetable}
  \mainitem{15}{Include directives}
  \\
  \inneritem{6}{Basic include}
  \\
	\inneritem{5}{Nested includes}
  \\
  \inneritem{4}{Include cycle detection}
  \\[4mm]

  \mainitem{15}{Define and undefine}
	\\
  \inneritem{4}{Basic macro substitution}
	\\
  \inneritem{3}{Undefine}
	\\
  \inneritem{4}{Nested replacement}
	\\
  \inneritem{4}{Cycle detection}
  \\[4mm]

  \mainitem{18}{Ifdefs}
  \\
  \inneritem{2}{Ifdef without else}
  \\
  \inneritem{2}{Ifdef with else}
  \\
  \inneritem{2}{Ifndef without else}
  \\
  \inneritem{2}{Ifndef with else}
  \\
  \inneritem{5}{Nested ifdefs, ifndefs}
  \\
  \inneritem{5}{Ifdef errors}
\end{gradetable}


\section{Submission}

Be sure to commit your source code and documentation to your
git repository, and to upload (push) those commits to the server
so that we may grade them.
Email the instructor or TA when your preprocessor is ready for grading.


\end{document}

