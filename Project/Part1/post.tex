
\documentclass{article}
\usepackage{fullpage}
\usepackage{url}
\usepackage{color}
\usepackage{listings}
\usepackage{longtable}

\renewcommand{\topfraction}{0.99}

\definecolor{dkgreen}{rgb}{0,0.6,0}
\definecolor{muave}{rgb}{0.58,0,0.82}
\definecolor{dkred}{rgb}{0.6,0,0}
\definecolor{dkblue}{rgb}{0,0,0.7}

\lstset{frame=none,
  language=C,
  % aboveskip=3mm,
  % belowskip=3mm,
  xleftmargin=2em,
  % xrightmargin=2em,
  showstringspaces=false,
  columns=flexible,
  basicstyle={\ttfamily},
  numbers=left,
  stepnumber=1,
  keywordstyle=\color{dkblue},
  directivestyle=\color{dkred},
  commentstyle=\color{dkgreen},
  stringstyle=\color{muave},
  breaklines=true,
  breakatwhitespace=true,
  tabsize=2
}

\lstdefinestyle{Output}{
  % aboveskip=3mm,
  % belowskip=3mm,
  xleftmargin=2em,
  % xrightmargin=2em,
  showstringspaces=false,
  columns=fixed,
  basicstyle={\ttfamily},
  numbers=none,
  keywordstyle=,
  directivestyle=,
  commentstyle=,
  stringstyle=,
  tabsize=8
}

\newcommand{\cpp}{C{\tt ++}}

\newcommand{\gradeline}{ \cline{1-2} \cline{4-4} ~\\[-1.5ex] }

\newenvironment{gradetable}{\begin{longtable}{@{}rrcp{5in}} \multicolumn{2}{l}{\bf Points} & & {\bf Description}\\ \gradeline}{\end{longtable}}

\newcommand{\mainitem}[2]{\pagebreak[2] {\bf #1} &&& {\bf #2}}
\newcommand{\mainpara}[1]{~ &&& {#1} }
\newcommand{\inneritem}[2]{~ & #1 && #2}
\newcommand{\innerpara}[1]{~ & ~ && #1}


\title{COM S 440/540 Project part 1}
\author{Create a lexer for C\thanks{Actually, it's a subset of C,
but students are always free to implement more of the C standard.}}
\date{}

\begin{document}

\maketitle

%======================================================================
\section{Overview}
%======================================================================

When executed with a mode of 1,
your compiler should read the specified input file
and divide it into tokens;
essentially this is the lexical analysis phase of the compiler
(referred to as the ``lexer'').
The output stream should contain
a line for each token,
showing the file name, line number, and text corresponding to the token.
More details (including the precise output and error formats,
and the definitions for the tokens) are given below.
The input file may be a C header file, a C source file,
or an arbitrary text file.
There are opportunities for extra credit.

All implementation must be C, C++, or Java.
Instructor permission is required to use anything beyond
the standard libraries for these languages.
Submissions will be graded on {\tt pyrite.cs.iastate.edu},
and therefore must build and run correctly there.

Students are \emph{strongly} encouraged
to encapsulate the functionality of this phase of the compiler,
so that later parts of the project can easily
examine and consume tokens from an input stream.

%======================================================================
\section{Tokens in our subset of C}
%======================================================================

Your lexer must recognize the tokens
given in Table~\ref{TAB:tokens}.
A useful list of integer constants for tokens may be found in {\tt tokens.h},
distributed with the materials for this part of the project.
Note that single-character symbols use their ASCII codes;
all other tokens have integer values above 300.
Whitespace (spaces, tabs, and carriage returns) serves only to separate
tokens, and should otherwise be discarded.
Any characters that are not part of a lexeme,
such as \verb|$|,
should generate an error message.

You may notice that some of the C keywords and operators are missing.
Students are welcome to implement additional language features if desired,
but any extra keywords or operators must be part ot the C standard.

\begin{table}[t]
	\centering
	\begin{tabular}{rlcrlcrl}
		\multicolumn{8}{c}{Single character symbols}
		\\[1mm]
		{\bf Value} & {\bf Lexeme}
		& ~~~~~~ &
		{\bf Value} & {\bf Lexeme}
		& ~~~~~~ &
		{\bf Value} & {\bf Lexeme}
		\\ \cline{1-2} \cline{4-5} \cline{7-8}
		33 & \verb|!| &&
		37 & \verb|%| &&
		38 & \verb|&| \\
		40 & \verb|(| &&
		41 & \verb|)| &&
		42 & \verb|*| \\
		43 & \verb|+| &&
		44 & \verb|,| &&
		45 & \verb|-| \\
		46 & \verb|.| &&
		47 & \verb|/| &&
		58 & \verb|:| \\
		59 & \verb|;| &&
		60 & \verb|<| &&
		61 & \verb|=| \\
		62 & \verb|>| &&
		63 & \verb|?| &&
		91 & \verb|[| \\
		93 & \verb|]| &&
		123 & \verb|{| &&
		124 & \verb+|+ \\
    125 & \verb|}| &&
    126 & \verb|~| \\
	\\
		\multicolumn{8}{c}{Operators with two characters}
		\\[1mm]
		{\bf Value} & {\bf Lexeme}
		& ~~~~~~ &
		{\bf Value} & {\bf Lexeme}
		& ~~~~~~ &
		{\bf Value} & {\bf Lexeme}
		\\ \cline{1-2} \cline{4-5} \cline{7-8}
		351 & \verb|==| &&
		352 & \verb|!=| &&
		353 & \verb|>=| \\
		354 & \verb|<=| &&
		355 & \verb|++| &&
		356 & \verb|--| \\
		357 & \verb+||+ &&
		358 & \verb|&&| &&
		361 & \verb|+=| \\
		362 & \verb|-=| &&
		363 & \verb|*=| &&
		364 & \verb|/=| \\
	\\
		\multicolumn{8}{c}{Keywords}
		\\[1mm]
		{\bf Value} & {\bf Lexeme}
		& ~~~~~~ &
		{\bf Value} & {\bf Lexeme}
		& ~~~~~~ &
		{\bf Value} & {\bf Lexeme}
		\\ \cline{1-2} \cline{4-5} \cline{7-8}
		401 & \verb|const| &&
		402 & \verb|struct| &&
		403 & \verb|for| \\
		404 & \verb|while| &&
		405 & \verb|do| &&
		406 & \verb|if| \\
		407 & \verb|else| &&
		408 & \verb|break| &&
		409 & \verb|continue| \\
		410 & \verb|return| &&
		411 & \verb|switch| &&
		412 & \verb|case| \\
		413 & \verb|default| \\
	\end{tabular}

	\begin{tabular}{rll}
		\\
		\multicolumn{3}{c}{Lexemes with attributes}
		\\[1mm]
		{\bf Value} & {\bf Lexeme} & {\bf Definition}
	  \\ \hline
		301 & type & \verb|void| or \verb|char| or \verb|int| or \verb|float|
		\\
		302 & character literal & a character enclosed in single quotes;
			escape sequences are extra credit
		\\
		303 & integer literal & a sequence of one or more digits
		\\
		304 & real literal & two sequences each of one or more digits, separated by \verb|.|
		\\
		305 & string literal & a sequence of zero or more characters (including escape sequences,
		\\ & & except for \verb|\\| and \verb|\"|) enclosed in double quotes
		\\
		306 & identifier & a letter or underscore, followed by zero or more letters,
			underscores, or
		\\ & & digits, that is not a keyword or type
	\end{tabular}

	\caption{Tokens to be recognized by the lexer.}
	\label{TAB:tokens}
\end{table}

\subsection{C style comments}

C-style comments, beginning with \verb|/*| and ending with \verb|*/|,
should be discarded and treated the same as a space.
A comment with no matching \verb|*/| is closed at the end of file,
and an error message
(indicating the starting point of the comment)
should be displayed for this case.
The formal rule for C-style comments is:
\begin{quote}
	When not currently inside a comment or string literal,
	the text \verb|/*| indicates the start of a comment,
	and all text is ignored until the next \verb|*/| or the end of file.
\end{quote}

\subsection{C++ style comments}

C++-style comments, beginning with \verb|//|
and ending with a newline or end of file,
should be discarded and treated the same as a newline character.
No error message is necessary if the comment is terminated by the end of file.
The formal rule for C++-style comments is:
\begin{quote}
	When not currently inside a comment or string literal,
	the text \verb|//| indicates the start of a comment,
	and all text is ignored until the next newline character or the end of file.
\end{quote}

\subsection{Literals}

The token definitions for integers, reals, characters, and strings
are simpler than the C standard.
Some extensions to these may be implemented for extra credit;
see below.


%======================================================================
\section{Formatting}
%======================================================================

\subsection{Compiler output}

For mode 1, the output stream should contain
a line for each token, exactly of the form
\begin{quote}
	{\tt File}
	\emph{filename}
	{\tt Line}
	\emph{line number}
	{\tt Token}
	\emph{token number}
	{\tt Text}
	\emph{lexeme}
\end{quote}
where
\begin{itemize}
	\item \emph{filename}
		is the name of the input file that contains the token,
	\item \emph{line number}
		is the line number of the input file that contains the token,
	\item \emph{token number}
		is the integer corresponding to the token, given in \verb|tokens.h|
		and Table~\ref{TAB:tokens}, and
	\item \emph{lexeme}
		is the matched text that produced the token.
\end{itemize}
Nothing else should be written to the output stream.

\subsection{Error messages}

Error messages generated by the lexer must
be written to standard error,
and have exactly the form
\begin{quote}
	\begin{tabbing}
		{\tt Lexer} \= {\tt error in file} \emph{filename}
		{\tt line} \emph{line number}
		{\tt near text} \emph{lexeme}
	\\
		\> \emph{Description}
	\end{tabbing}
\end{quote}
where the ``near text \emph{lexeme}'' portion is optional
(for cases where it does not make sense).
The \emph{Description} should give a helpful description of the error,
and may span more than one line if necessary.
Warning messages should have the same format,
except with ``{\tt warning}'' in place of ``{\tt error}''.
The error stream must be empty if no errors or warnings are generated.
Students are \emph{strongly} encouraged to define their own switches
	to display any additional information, to help with debugging.


Example inputs, outputs, and errors are given below.
Additionally, students are encouraged to check the output and error messages
given for the example test files, for all parts of the project.
For test files with errors,
code is considered correct if the first error messages given
occur on the same input line and the error messages describe the
same error (as determined by a human).

%======================================================================
\section{Limits}
%======================================================================

You may set resonable limits,
at least as large as specified below, for lexemes.
\begin{itemize}
	\item Integer literals: 48 characters
	\item Real literals: 48 characters
	\item Identifiers: 48 characters
	\item String literals: 1024 characters
\end{itemize}
Your lexer may terminate with an appropriate error message
if these limits are exceeded.
Note that there is {\bf no limit} for
the length of a line,
the length of a comment,
or the length of an input file.

%======================================================================
\section{Extra credit options}
%======================================================================

\subsection{Character literals}

In addition to ordinary character literals,
e.g., \verb|' '|,
your lexer should allow
the escape sequences
\verb|'\a'|, \verb|'\b'|, \verb|'\n'|, \verb|'\r'|, \verb|'\t'|, \verb|'\\'|,
and \verb|'\''|.

\subsection{Integer literals}

In addition to decimal (base ten) integer literals,
your lexer should allow \emph{octal} literals and \emph{hexadecimal} literals.
The rules\footnote{This is still simpler than the actual standard.} are
\begin{itemize}
	\item An octal literal starts with \verb|0|,
				followed by zero or more digits \verb|0| through \verb|7|.
				An error should be generated for digits \verb|8|, \verb|9|,
				\verb|a| through \verb|f|, or \verb|A| through \verb|F|.
	\item A hexadecimal literal starts with \verb|0x| or \verb|0X|,
		followed by zero or more digits \verb|0| through \verb|9|,
		\verb|a| through \verb|f|, or \verb|A| through \verb|F|.
	\item A decimal literal is
				a sequence of one or more digits \verb|0| through \verb|9|
				that does not start with \verb|0|.
				An error should be generated for digits
				\verb|a| through \verb|f|, or \verb|A| through \verb|F|.
\end{itemize}
Note that the literal 0 is technically octal in this definition,
but since 0 is the same in octal and decimal, this isn't a problem.

\subsection{Real literals}

In addition to numbers with fractional portions,
e.g., \verb|3.14159|,
your lexer should allow \emph{exponents},
of the form \verb|e| or \verb|E| followed by an integer with optional sign.
If an exponent is present, then the fractional part is optional.
For example,
the following should be recognized as real literals:
\begin{verbatim}
	31.41592
	6.02214e23
	1.60217653E-19
	1e+100
\end{verbatim}
You may assume that all real literals are decimal,
and generate a lexer error for real literals starting
with \verb|0x| or \verb|0X|.

\subsection{String literals}

Allow escape sequences \verb|\\| and \verb|\"|,
in addition to the others, inside string literals.
For example, the following should be recognized as string literals:
\begin{verbatim}
	"Hello, world!\n"
	"I said, \"Hello, world!\""
	"This is evil \\"
\end{verbatim}

\subsection{Lexeme truncation}

If your lexer encounters a lexeme that is too long,
it should \emph{truncate} it
(and generate an appropriate warning message)
by ignoring extra characters
and continuing to process the input stream.

%======================================================================
\section{Basic example: {\tt hello.c}}
%======================================================================

\subsection{Input}

\lstinputlisting{hello.c}

\subsection{Output for {\tt mycc -1 hello.c}}

\lstinputlisting[style=Output]{hello.c.min.out.txt}

\subsection{Errors for {\tt mycc -1 hello.c}}

\lstinputlisting[style=Output]{hello.c.min.err.txt}


%======================================================================
\section{Trickier example: {\tt tricky.c}}
%======================================================================

\subsection{Input}

\lstinputlisting{tricky.c}

\subsection{Output for {\tt mycc -1 tricky.c}}

\lstinputlisting[style=Output]{tricky.c.min.out.txt}

\subsection{Errors for {\tt mycc -1 tricky.c}}

\lstinputlisting[style=Output]{tricky.c.min.err.txt}


%======================================================================
\section{Extra credit example: {\tt extra.c}}
%======================================================================

\subsection{Input}

\lstinputlisting{extra.c}

\subsection{Sample output when extra credit is not implemented}

\lstinputlisting[style=Output]{extra.c.min.out.txt}

\subsection{Sample errors when extra credit is not implemented}

\lstinputlisting[style=Output]{extra.c.min.err.txt}

\subsection{Sample output when extra credit is implemented}

\lstinputlisting[style=Output]{extra.c.max.out.txt}

\subsection{Sample errors when extra credit is implemented}

\lstinputlisting[style=Output]{extra.c.max.err.txt}



%======================================================================
\section{Grading}
%======================================================================

\noindent
\begin{gradetable}
  \mainitem{20}{Documentation}
  \\[1mm]
  \inneritem{5}{\tt README.txt}
  \\[1mm]
  \innerpara{%
    How to build the compiler and documentation.
    Updated to show which part 1 features are implemented.
  }
  \\[1mm]
  \inneritem{15}{\tt developers.pdf}
  \\[1mm]
  \innerpara{%
    New section for part 1, that explains
    the purpose of each source file,
    the main data structures used,
    and gives a high-level overview of how the various
    features are implemented.
  }
  \\[4mm]

  \mainitem{10}{Ease of building}
  \\[1mm]
  \mainpara{%
    How easy was it for the graders to build your compiler and
    documentation from the {\tt README} file.
		You are encouraged to use {\tt Makefile}s under the
		{\tt Source/} and {\tt Documentation/}
		directories so that running ``{\tt make}''
		will build everything and running ``{\tt make clean}''
		will remove all generated files.
  }
  \\[4mm]

  \mainitem{5}{Still works in mode 0}
  \\[4mm]

  \mainitem{5}{The {\tt -o} switch works}
  \\[4mm]

  \mainitem{60}{Basic lexer}
  \\
  \inneritem{10}{Correct line numbers and output format}
  \\
	\inneritem{15}{Keywords, types, identifiers}
  \\
  \inneritem{12}{Integer, real, string, character literals}
  \\
  \inneritem{2}{Error for invalid characters}
  \\
  \inneritem{6}{Comments}
  \\
  \inneritem{15}{Symbols}
  \\[4mm]

	\mainitem{20}{Extra credit}
	\\
	\inneritem{4}{Character literals with escapes}
	\\
	\inneritem{4}{Octal, Decimal, Hexadecimal integer literals}
	\\
	\inneritem{4}{Real literals with exponents}
	\\
	\inneritem{4}{String literals with escapes}
	\\
	\inneritem{4}{Truncated lexemes}
	\\

  \gradeline
  \mainitem{100}{Total for students in 440 (max points is 110)}
  \\
  \mainitem{112}{Total for students in 540}
\end{gradetable}


\section{Submission}

Be sure to commit your source code and documentation to your
git repository, and to upload (push) those commits to the server
so that we may grade them.
In Canvas,
indicate if you would like us to re-grade your part 0 submission
for reduced credit, or only grade part 1.
If nothing is indicated, we will grade part~1 only.


\end{document}

