
\documentclass{article}
\usepackage{fullpage}
\usepackage{hyperref}
\usepackage{color}
\usepackage{listings}
\usepackage{longtable}

\definecolor{dkgreen}{rgb}{0,0.6,0}
\definecolor{muave}{rgb}{0.58,0,0.82}
\definecolor{dkred}{rgb}{0.6,0,0}
\definecolor{dkblue}{rgb}{0,0,0.7}

\lstset{frame=none,
  language=C,
  % aboveskip=3mm,
  % belowskip=3mm,
  xleftmargin=2em,
  % xrightmargin=2em,
  showstringspaces=false,
  columns=flexible,
  basicstyle={\ttfamily},
  numbers=left,
  keywordstyle=\color{dkblue},
  directivestyle=\color{dkred},
  commentstyle=\color{dkgreen},
  stringstyle=\color{muave},
  breaklines=true,
  breakatwhitespace=true,
  tabsize=2
}

\lstdefinestyle{jvm}{
  % aboveskip=3mm,
  % belowskip=3mm,
  xleftmargin=2em,
  % xrightmargin=2em,
  showstringspaces=false,
  columns=flexible,
  basicstyle={\ttfamily},
  numbers=left,
  moredelim=[s][\color{black}]{L}{;},
  morecomment=[l][\color{dkgreen}]{;},
  morecomment=[l][\color{magenta}]{;;},
  keywords={class,public,static,super,method,code,end},
  keywordstyle=\color{dkblue},
  breaklines=true,
  breakatwhitespace=true,
  tabsize=8
}


\newcommand{\gradeline}{ \cline{1-2} \cline{4-4} ~\\[-1.5ex] }

\newenvironment{gradetable}{\begin{longtable}{@{~~}rrcp{5in}} \multicolumn{2}{l}{\bf Points} & & {\bf Description}\\ \gradeline}{\end{longtable}}

\newcommand{\mainitem}[2]{\pagebreak[2] {\bf #1} &&& {\bf #2}}
\newcommand{\mainpara}[1]{~ &&& {#1} }
\newcommand{\inneritem}[2]{~ & #1 && #2}
\newcommand{\innerpara}[1]{~ & ~ && #1}

\newcommand{\markedinneritem}[2]{\multicolumn{1}{@{}l}{$\dagger$} & #1 && #2}

\newcommand{\parser}{3}
\newcommand{\typecheck}{4}
\newcommand{\codegen}{5}
\newcommand{\flowgen}{6}


\title{COM S 440/540 Project part \codegen}
\author{Code generation: expressions}
\date{}

\newenvironment{quotewide}[1]{\begin{minipage}[b]{#1} \vspace{1mm} \begin{quote}} {\end{quote} \vspace{1mm} \end{minipage}}

\begin{document}

\maketitle

%======================================================================
\section{Requirements for part \codegen}
%======================================================================

When executed with a mode of \codegen,
your compiler should read the specified input file,
and check it for correctness (including type checking) as done in part~\typecheck.
If there are no errors, then your compiler should output
an equivalent program in our target language
(see Section~\ref{SEC:target}).
For this part of the project,
your compiler must generate correct code for only some types of
statements:
namely, expressions (except for the ternary operator),
including function calls and returns.
Control flow (conditionals, the ternary operator, and loops)
will be required for the next and final part of the project.
As usual,
any error messages should be written to standard error,
and your compiler may make a ``best effort'' to continue processing
the input file, or exit.
Errors detected in this part of the project should have format
\begin{quote}
  \begin{tabbing}
		{\tt Code ge}\={\tt neration error in file }\emph{filename}
		{\tt line }\emph{line number}
	\\
		\> \emph{Description}
  \end{tabbing}
\end{quote}
where line numbers will be specified below if not obvious.




%======================================================================
\section{I/O}
%======================================================================

To manage I/O,
for this and the next part of the project,
you should assume that our source language has
the following \emph{built-in} functions
(i.e., you should add them to your symbol table before compilation starts).
\begin{itemize}
  \item \lstinline|int getchar()|:
    This should behave the same as {\tt getchar()} in {\tt stdio.h}.

  \item \lstinline|int putchar(int c)|:
    This should behave the same as {\tt putchar()} in {\tt stdio.h}.
\end{itemize}
These functions will be implemented in a supporting library file,
{\tt libc.class}, that will be placed in the same directory
as your assembled class file.

Most C compilers will correctly compile source inputs
that call {\tt getchar()} and {\tt putchar()}
without including {\tt stdio.h};
this will allow you to test your compiler against any production C compiler
on the same input programs.
Alternatively,
you can add prototypes for {\tt getchar()} and {\tt putchar()}
at the beginning of your test code,
assuming your compiler correctly handles
multiple prototypes for the same function (as it should).

%======================================================================
\section{The target language} \label{SEC:target}
%======================================================================

The target language is Java assembly language,
as read by the Krakatau Java assembler
(written in Python and available on github at
\url{https://github.com/Storyyeller/Krakatau}).
You will build assembly code for a class whose name
matches the source file being compiled,
with global variables becoming static members,
and functions becoming static methods.
The produced class should be derived from {\tt java/lang/Object}.

Blank lines may be added to improve readability of the output file.
Comments,
which begin with a semi-colon and end with a newline character,
may appear on an empty line or at the end of a valid line.
Students are encouraged to generate comments to help debug their compilers.

Global variables may appear in output in any order
(recommended is to output them as encountered)
as long as they appear before they are used.

Your compiler {\bf may not} invoke the Java compiler.
However, it is perfectly fine to ``reverse engineer'' the Java compiler,
to help you figure out which assembly instructions to use for translation.
For instance, you could write Java code, compile it, then use the Krakatau
disassembler to view the assembly code generated by the Java compiler,
to help you learn and understand the Java bytecode instructions.

\subsection{User-defined functions}

Each function with a body in the C source
should cause the generation of (Java assembly for)
a static method.
User-defined methods should appear before the special ones (see below).
It is recommended, but not required,
to generate assembly for functions in the same order as
they appear in the input file,
and as they are encountered.

\subsection{Java's main method}

Your compiler should produce (Java assembly for) a Java {\tt main()}
method.
In Java source, the method would have prototype
\begin{quote}
  \begin{lstlisting}[numbers=none]
    public static void main(String args[]);
  \end{lstlisting}
\end{quote}
and its job is to invoke the static {\tt main()} method
and to display the returned integer
(making it easier to test your compiler).
Assume the C main method has prototype
\begin{quote}
  \begin{lstlisting}[numbers=none]
    int main()
  \end{lstlisting}
\end{quote}
and use the format
\begin{quote}
  \tt Return code: \emph{code}
\end{quote}
to display the returned integer from C's {\tt main()} function.


\subsection{Special method {\tt <init>}}

You will need to output (Java assembly for) the special method {\tt <init>}
to construct an instance of the class.
For the generated code,
this method does not need to do anything
except invoke the constructor for base class {\tt java/lang/Object}.

\subsection{Special method {\tt <clinit>}}

The special method {\tt <clinit>}
is used to initialize any static members that require initialization.
For the C source code,
this includes global arrays and global variables that were initialized
when declared (this was extra credit in parts \parser{} and \typecheck{}).
If there are no such items to initialize, then method {\tt <clinit>}
may be omitted.







\subsection{Annotations}
\label{SEC:annotations}

Special comments that begin with a double semi-colon
and appear on their own line
are used to annotate the assembly code
for use with a utility that we will use to check your code
  (see Section~\ref{SEC:checking}).
All annotations should end with the \emph{location}
in the source (the file name and line number)
that caused the annotation.
The following annotations should be used.

\paragraph{Parameters and local variables.}
For each user-defined function that appears in the generated assembly code,
the name and ``address''
(non-negative index) for each parameter and local variable
should be given
somewhere in the function body, before its first use.
The format should be
``parameter'' or ``local'',
followed by the address,
followed by the name,
followed by the location.
The JVM requires that parameters have addresses that start from 0 and
correspond to parameter order,
and local variables have addresses following the parameters.
For example, the C function
\begin{lstlisting}
  int foo(int a, int b)
  {
    int c, d;
    return 0;
  }
\end{lstlisting}
could generate annotations
\begin{lstlisting}[style=jvm]
  ;; parameter 0 a foo.c 1
  ;; parameter 1 b foo.c 1
  ;; local 2 c foo.c 3
  ;; local 3 d foo.c 3
\end{lstlisting}
where parameter {\tt a} \emph{must} have address 0,
parameter {\tt b} \emph{must} have address 1,
and parameters {\tt c} and {\tt d} can have any addresses 2 or higher.


\paragraph{Expressions and return statements.}
For each top-level expression statement or return statement,
an annotation should be given with the input file name and line number
where the expression appears in the source file.
The format should be
``expression'' or ``return'',
followed by the location.
The annotation should appear immediately before the code generated
to implement the expression or return statement.
For example, the C function
\lstinputlisting{foo.c}
might generate the following annotated code.
\lstinputlisting[style=jvm,firstline=5,lastline=20]{foo.smart.j}


%======================================================================
\section{Checking your generated code} \label{SEC:checking}
%======================================================================

\subsection{Test by running}

You should be able to assemble the code generated by your compiler
(using the Krakatau assembler)
to obtain a class file.
You can then run this class file,
just as if it were compiled from Java source.
The script {\tt CheckR\_5.sh} is based on this idea:
\begin{enumerate}
  \item
  It runs your compiler with mode {\tt -\codegen} on the C source code,
    and produces a {\tt .j} file containing the Java assembly code.

  \item
  It runs the assembler on the {\tt .j} file produced by your compiler.

  \item
  It  runs the resulting {\tt .class} file on a JVM,
    with one or more input files (named {\tt .input1}, etc)
  and checks the output against the expected output.
\end{enumerate}
The expected outputs are obtained using {\tt gcc} to compile
the C source
(modified to print the value returned by {\tt main()}).
You can generate your own test programs and inputs,
using a C file {\tt base.c},
and
inputs {\tt base.input001}, {\tt base.input002}, etc.
Then, use {\tt CheckR\_5.sh -G base.c}
to automatically generate the ``correct'' outputs.
Use {\tt CheckR\_5.sh "run your compiler" base.c},
where the string {\tt "run your compiler"}
is replaced by the command used to run your compiler,
to test your compiler.

\subsection{Compare assembly code or analyzed expressions}

The script {\tt CheckJ\_5.sh} compares your compiler output against
the instructor output.
The {\tt -j} switch causes the script to compare raw assembly.
However, different assembly instructions may still produce correct code.
Running with {\tt -x} or with no switch causes the script to analyze
what is computed on the JVM stack using the {\tt jexpr} utility.
Output is compared for each annotated line.

\subsection{The {\tt jexpr} utility}

The {\tt jexpr} utility
(source code available in the course repository)
examines Java assembly code and runs it,
keeping track of what is computed.
This can help you to debug stack errors,
and gives a somewhat more readable summary of what each function computes.
See Section~\ref{SEC:examples} for example output.



%======================================================================
\section{Examples} \label{SEC:examples}
%======================================================================

\subsection{Input: {\tt sum.c}}

\lstinputlisting{sum.c}

\subsection{Output: {\tt sum.j}}

\lstinputlisting[style=jvm]{sum.basic.j}

\subsection{Output with smart stack management (extra credit)}
\label{SEC:smartstack}

\lstinputlisting[style=jvm]{sum.smart.j}

\subsection{Output of {\tt jexpr} on \ref{SEC:smartstack} code}

\lstinputlisting[style=jvm]{sum.smart.jexpr}

%======================================================================
\section{Grading} \label{SEC:grading}
%======================================================================

For all students: implement as many or as few features listed below as you wish,
but keep in mind that some features will make testing your code \emph{much}
easier (features needed to test your code for part~\flowgen{}
are marked with $\dagger$),
and a deficit of points will impact your overall grade.
Excess points will count as extra credit.


\noindent
\begin{gradetable}
  \mainitem{15}{Documentation}
  \\[1mm]
  \inneritem{3}{\tt README.txt}
  \\[1mm]
  \innerpara{%
    How to build the compiler and documentation.
    Updated to show which part \codegen{} features are implemented.
  }
  \\[1mm]
  \inneritem{12}{\tt developers.pdf}
  \\[1mm]
  \innerpara{%
    New section for part \codegen{}, that explains
    the purpose of each source file,
    the main data structures used (or how they were updated),
    and gives a high-level overview of how the target code
    is generated.
  }
  \\[4mm]

  \mainitem{10}{Ease of grading}
  \\[1mm]
  \inneritem{4}{Building}
  \\[1mm]
  \innerpara{%
    How easy was it for the graders to build your compiler and
    documentation?
    For full credit, simply running ``{\tt make}''
    should build both the documentation and the compiler executable,
    and running ``{\tt make clean}'' should remove
    all generated files.
  }
  \\[1mm]
  \inneritem{6}{Output and formatting}
  \\[1mm]
  \innerpara{%
    Assembly code is annotated as discussed in Section~\ref{SEC:annotations},
    so it can be parsed and analyzed by grading scripts.
  }
  \\[4mm]

  \mainitem{10}{Still works in modes 0 through \typecheck}
  \\[4mm]

  \mainitem{10}{Always present output}
  \\[1mm]
  \markedinneritem{3}{Class with proper name; super}
  \\[1mm]
  \markedinneritem{3}{Special method {\tt <init>}}
  \\[1mm]
  \markedinneritem{4}{Java {\tt main()}, calls C {\tt main()} and shows return value}
  \\[4mm]

  \mainitem{10}{Code for user functions}
  \\[1mm]
  \markedinneritem{3}{Correct parameters and return type}
  \\[1mm]
  \markedinneritem{2}{Correct {\tt .method} and {\tt .code} blocks}
  \\[1mm]
  \markedinneritem{3}{Reasonable stack limit}
  \\[1mm]
  \markedinneritem{2}{Correct local count}
  \\[4mm]

  \mainitem{15}{Function calls and returns}
  \\[1mm]
  \markedinneritem{4}{Parameter set up}
  \\[1mm]
  \inneritem{3}{Function call}
  \\[1mm]
  \markedinneritem{4}{Correct calls to built-ins {\tt getchar} and {\tt putchar}}
  \\[1mm]
  \inneritem{4}{Void, char, int, float returns}
  \\[4mm]

  \mainitem{10}{Expressions: literals, variables}
  \\[1mm]
  \markedinneritem{3}{Character, integer, and float literals}
  \\[1mm]
  \markedinneritem{3}{Reading local variables and parameters}
  \\[1mm]
  \inneritem{4}{Reading global variables}
  \\[4mm]

  \mainitem{15}{Operators}
  \\[1mm]
  \markedinneritem{10}{Binary operators {\tt +}, {\tt -}, {\tt *}, {\tt /}, {\tt \%}}
  \\[1mm]
  \inneritem{5}{Unary operators and type conversions}
  \\[4mm]

  \mainitem{15}{Global variable, local variable, and parameter writes}
  \\[1mm]
  \inneritem{3}{Local variable initialization}
  \\
  \innerpara{%
    Requires variable initialization support,
    which was extra credit for parts~\parser{} and \typecheck{}.
  }
  \\[1mm]
  \markedinneritem{4}{Assignment expressions with {\tt =}}
  \\[1mm]
  \inneritem{4}{Update assignments: {\tt +=}, {\tt -=}, {\tt *=}, {\tt /=}}
  \\[1mm]
  \inneritem{4}{Pre and post increment and decrement}
  \\[4mm]

  \mainitem{18}{Arrays}
  \\[1mm]
  \inneritem{3}{Local array allocation}
  \\[1mm]
  \inneritem{3}{Reading array elements in expressions}
  \\[1mm]
  \inneritem{3}{Array element assignments with {\tt =}}
  \\[1mm]
  \inneritem{3}{Array element updates: {\tt +=}, {\tt -=}, {\tt *=}, {\tt /=}}
  \\[1mm]
  \inneritem{3}{Passing arrays as parameters}
  \\[1mm]
  \inneritem{3}{Passing string literals as {\tt char[]} parameters}
  \\[4mm]

  \mainitem{10}{Special method {\tt <clinit>}}
  \\[1mm]
  \inneritem{4}{Allocates global arrays}
  \\[1mm]
  \inneritem{4}{Initializes global variables}
  \\
  \innerpara{%
    Requires variable initialization support,
    which was extra credit for parts~\parser{} and \typecheck{}.
  }
  \\[1mm]
  \inneritem{2}{Method is present when needed, omitted when not needed}
  \\[4mm]

  \mainitem{4}{Smart stack management}
  \\[1mm]
  \mainpara{%
    Avoid saving assignment, update, and increment results on
    the stack for cases where those values will be popped off
    and discarded.
  }
  \\[4mm]

  \mainitem{4}{Missing return statements}
  \\[1mm]
  \mainpara{%
    Checks for missing return statements.
    Gives an error (with a line number as close as possible
    to the end of the function body) for non-void functions
    without a return statement.
    Adds a return statement if needed for void functions.
  }
  \\[4mm]

  \mainitem{4}{Dead code elimination}
  \\[1mm]
  \mainpara{%
    Statements following a return statement
    are eliminated.
  }
  \\[4mm]

  \gradeline
  \mainitem{100}{Total for students in 440 (max points is 120)}
  \\
  \mainitem{120}{Total for students in 540 (max points is 140)}
\end{gradetable}


%======================================================================
\section{Submission}
%======================================================================

\begin{table}[h]
\centering

  \begin{tabular}{lcr}
    {\bf Part} & ~~~~ & {\bf Penalty applied} \\ \cline{1-1}\cline{3-3}
    Part 0 && 40\% off \\
    Part 1 && 30\% off \\
    Part \parser{} && 20\% off \\
    Part \typecheck{} && 10\% off
  \end{tabular}

\caption{Penalty applied when re-grading}
\label{TAB:penalties}
\end{table}

Be sure to commit your source code and documentation to your
git repository, and to upload (push) those commits to the server
so that we may grade them.
In Canvas,
indicate which parts you would like us to re-grade for reduced credit
(see Table~\ref{TAB:penalties} for penalty information).
Otherwise, we will grade only part \codegen.




\end{document}
