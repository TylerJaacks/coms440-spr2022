%
% LaTeX source file.
%
% To build a pdf, use 'pdflatex post.tex' on most systems.
%

\documentclass{article}
\usepackage{fullpage}
\usepackage{url}
\usepackage{longtable}
\usepackage{color}
\usepackage{listings}

\lstset{frame=none,
  language=C,
  % aboveskip=3mm,
  % belowskip=3mm,
  xleftmargin=2em,
  % xrightmargin=2em,
  showstringspaces=false,
  columns=flexible,
  basicstyle={\ttfamily},
  numbers=none,
  keywordstyle=,
  directivestyle=,
  commentstyle=,
  stringstyle=,
  tabsize=8
}


\title{COM S 440/540 Project part 0}
\author{Start the compiler}
\date{}

\newcommand{\compiler}{mycc}

\newcommand{\gradeline}{ \cline{1-2} \cline{4-4} ~\\[-1.5ex] }

%\newenvironment{gradetable}{\begin{tabular}{@{}rrcl} \multicolumn{2}{l}{\bf Points} & & {\bf Description}\\ \gradeline}{\end{tabular}}

\newenvironment{gradetable}{\begin{longtable}{@{}rrcp{5in}} \multicolumn{2}{l}{\bf Points} & & {\bf Description}\\ \gradeline}{\end{longtable}}

\newcommand{\mainitem}[2]{\pagebreak[2] {\bf #1} &&& {\bf #2}}
\newcommand{\mainpara}[1]{~ &&& {#1} }
\newcommand{\inneritem}[2]{~ & #1 && #2}
\newcommand{\innerpara}[1]{~ & ~ && #1}


\begin{document}

\maketitle

\section{Overview}

\begin{enumerate}
	\item
	Create a project at
	\url{https://git.ece.iastate.edu}.
	\begin{itemize}
		\item
    Set the ``Project Visibility'' to ``Private''.

		\item
    Add the Instructor and TA
			(usernames {\tt asminer} and {\tt rdjiles})
    as ``Developers'',
			with an expiration date of June 30, \the\year.

    \item
    You will use the same git repository for all parts of the project,
    for the entire semester.

		\item
		You should use the directory structure shown in Table~\ref{TAB:dirs}.

  \end{itemize}

	\item
	Add all necessary source files, including the {\tt Makefile}
	and {\tt README.txt}/{\tt README.md}  file, to the git repository.
	{\bf DO NOT} add any files that are automatically generated
	from source files.

	\item
	Ensure that all changes are committed and pushed to the server.

	\item
	Submit in Canvas.

\end{enumerate}

\begin{table}[t]
	\centering
		\begin{tabular}{lcl}
			{\bf Name} & & {\bf Purpose}
			\\ \cline{1-1} \cline{3-3}
			\\[-1.6ex]
			{\tt Documentation/} & & for \LaTeX{} sources of project documentation
			\\[.2ex]
			{\tt Grading/} & & for Instructor and TA feedback (put a short
													README file inside)
			\\[.2ex]
			{\tt Source/} & & for your source code, which may be further
												subdivided as you wish
		\end{tabular}
	\caption{Directory hierarchy to use within your git repository.}
	\label{TAB:dirs}
\end{table}

\section{The compiler executable}

You must implement your compiler in C, C++, or Java (or a combination),
and your {\tt Makefile} should produce an executable (or jar file)
named ``\compiler''.
This will be a command-line compiler, so all interaction will be done
using arguments and switches on the command line.
The compiler should \emph{never} prompt the user for anything.
Any messages (warnings, errors, or information)
not related to the required output
should be written to standard error.
You may abort execution, or valiantly attempt to continue,
after the first error message.

The compiler should be invoked using the following
command-line arguments and switches, discussed below.
\begin{quote}
\tt \compiler{} -mode [-o outfile] infile
\end{quote}
The ``{\tt []}'' notation indicates that an item is optional.
As a special case, though,
if the compiler is invoked with no arguments at all,
  you should display usage information to standard error.
If you choose, you may implement a more permissive compiler
(e.g., maybe you allow the mode or output file switches
to appear in any order).
You may implement additional switches if you like;
for example,
  you might add switches to display information
  to help you with debugging.

\subsection{The mode switch}

The mode is an integer from 0 to 5,
and corresponds to each part of the project.
We will build up the compiler in pieces,
each one roughly corresponding to one of the compiler phases.
This allows students to work ahead and not worry about disturbing the graders
(e.g., you can work on part 3 while we are grading part 2,
because we will test with {\tt -2} and you will test with {\tt -3}).
Equally importantly,
it allows for regression testing:
because part 3 will build upon part 2,
if a test input file fails for part 3,
  you might want to make sure that your part 2 solution can handle it.

\subsection{Input files}

For most modes, it is an error if no input files are specified;
your compiler should display an appropriate error message
(to standard error, of course) in this case.
Generally, you will need to process a single input file.
Students may choose to process more than one input file.

\subsection{Compiler output}

Each part of the project will require a different form of output,
in a specified format.
If no output file is specified, then the output should be written
to standard output.
Otherwise, if the {\tt -o} switch specifies an output file,
then output should be written to the specified file.

\subsection{Error messages}

{\bf All error messages} for all parts of the project
should be written to {\bf standard error}.
When errors are caused by the input file (e.g., a syntax error),
the error message should display the filename, line number, and text
(as appropriate) that caused the error.
This will be specified further in later parts of the project.

\subsection{Some assumptions}


You may safely assume:
\begin{itemize}
  \item When grading part $n$ of the project,
        your compiler will be invoked with
        first argument ``{\tt -}$n$'',
        and no other modes will be specified on the command line.

  \item If an output file is specified,
        it will appear directly after the mode.

  \item Output file names, if specified,
        will never begin with ``{\tt -}''.

  \item Input file names will appear last
        (after the mode and after the output file name, if present)
        and will never begin with ``{\tt -}''.

\end{itemize}


\section{Requirements for part 0}

\begin{itemize}
  \item If no arguments are given, display usage information to standard error.
  \item For mode 0, ignore any input files
    (you may give a warning about this)
    and display, to the output stream,
    the following information
    (this will help with grading):
    \begin{itemize}
      \item The compiler name
      \item Author and contact information
      \item A version number and a date of ``release''
    \end{itemize}

\end{itemize}

\section{Examples}


Running
\begin{verbatim}
    mycc
\end{verbatim}
should display something like the following on standard error.
\lstinputlisting{noargs.out.txt}
%
%
Running
\begin{verbatim}
    mycc -0
\end{verbatim}
should produce output like the following (to standard output).
\lstinputlisting{zeroarg.out.txt}
%
%
Running
\begin{verbatim}
    mycc -0 -o out.txt
\end{verbatim}
should produce the same output in file {\tt out.txt}.

\section{Grading}

\noindent
\begin{gradetable}
  \mainitem{15}{Documentation}
  \\[2mm]
  \inneritem{5}{\tt README.txt} or {\tt README.md}
  \\[2mm]
  \innerpara{%
    Explains how to build the compiler and documentation.
    For each part of the project,
    explains which features are implemented.
  }
  \\[2mm]
  \inneritem{10}{\tt developers.pdf}
  \\[2mm]
  \innerpara{%
    Built from the \LaTeX{} source {\tt developers.tex}.
    Add a new section to this document for each part of the project.
    This should explain, to other developers (students in the course),
    the purpose of each source file,
    the main data structures used,
    and give a high-level overview of how your code works.
    (Note: this is really overkill for this part of the project;
    I do not expect much depth here yet.)
  }
  \\[5mm]

  \mainitem{10}{Ease of building}
  \\[2mm]
  \mainpara{%
    How easy was it for the graders to build your compiler and
    documentation from the {\tt README} file.
    For full credit, simply running ``{\tt make}''
    should build both the documentation and the compiler executable,
    and running ``{\tt make clean}'' should remove
    all generated files.
  }
  \\[5mm]

  \mainitem{15}{Working executable (test on {\tt pyrite.cs.iastate.edu})}
  \\[2mm]
  \inneritem{5}{Correct behavior for {\tt -0}}
  \\[2mm]
  \innerpara{%
    Version and author information is written to output.
  }
  \\[3mm]
  \inneritem{5}{The {\tt -o} switch works}
  \\[2mm]
  \innerpara{%
    Using {\tt -o} causes output
    to go to the specified file.
  }
  \\[3mm]
  \inneritem{5}{Correct behavior for no arguments}
  \\[2mm]
  \innerpara{%
    Usage information written to standard error.
  }
  \\[3mm]
  \gradeline
  \mainitem{40}{Total for students in 440}
  \\
  \mainitem{40}{Total for students in 540}
\end{gradetable}


\section{Submission in Canvas}

In Canvas, submit the URLs to clone your git repository via ssh and https.
We will grade your work by cloning a copy of your repository
on {\tt pyrite.cs.iastate.edu}, and testing there.
Students are strongly encouraged to test on {\tt pyrite} themselves.



\end{document}

