
\documentclass{article}
\usepackage{fullpage}
\usepackage{url}
\usepackage{color}
\usepackage{listings}
\usepackage{longtable}

\renewcommand{\topfraction}{0.99}

\definecolor{dkgreen}{rgb}{0,0.6,0}
\definecolor{muave}{rgb}{0.58,0,0.82}
\definecolor{dkred}{rgb}{0.6,0,0}
\definecolor{dkblue}{rgb}{0,0,0.7}

\lstset{frame=none,
  language=C,
  % aboveskip=3mm,
  % belowskip=3mm,
  xleftmargin=2em,
  % xrightmargin=2em,
  showstringspaces=false,
  columns=flexible,
  basicstyle={\ttfamily},
  numbers=left,
  stepnumber=1,
  keywordstyle=\color{dkblue},
  directivestyle=\color{dkred},
  commentstyle=\color{dkgreen},
  stringstyle=\color{muave},
  breaklines=true,
  breakatwhitespace=true,
  tabsize=2
}

\lstdefinestyle{Output}{
  % aboveskip=3mm,
  % belowskip=3mm,
  xleftmargin=2em,
  % xrightmargin=2em,
  showstringspaces=false,
  columns=flexible,
  basicstyle={\ttfamily},
  numbers=none,
  keywordstyle=,
  directivestyle=,
  commentstyle=,
  stringstyle=,
  tabsize=2
}

\newcommand{\gradeline}{ \cline{1-2} \cline{4-4} ~\\[-1.5ex] }

\newenvironment{gradetable}{\begin{longtable}{@{}rrcp{5in}} \multicolumn{2}{l}{\bf Points} & & {\bf Description}\\ \gradeline}{\end{longtable}}

\newcommand{\mainitem}[2]{\pagebreak[2] {\bf #1} &&& {\bf #2}}
\newcommand{\mainpara}[1]{~ &&& {#1} }
\newcommand{\inneritem}[2]{~ & #1 && #2}
\newcommand{\innerpara}[1]{~ & ~ && #1}

\newcommand{\iseasier}{is$^\dagger$ }
\newcommand{\isextra}{is$^\ddagger$ }

\newcounter{rule}

\newcommand{\rulenumber}[1]{\refstepcounter{rule}R\therule\label{RULE:#1}}

\newcommand{\parser}{3}
\newcommand{\typecheck}{4}
\newcommand{\codegen}{5}


\title{COM S 440/540 Project part \typecheck}
\author{Type checking}
\date{}

\begin{document}

\maketitle


%======================================================================
\section{Requirements for part \typecheck}
%======================================================================

When executed with a mode of \typecheck,
your compiler should read the specified input file
and check that the file has correct C syntax
(as done in part~\parser),
and perform type checking on all expressions.
For any type errors (discussed in Section~\ref{SEC:types}),
write an error message to standard error, in the format
\begin{quote}
  \begin{tabbing}
		{\tt Type ch}\={\tt ecking error in file }\emph{filename}
		{\tt line }\emph{line number}
	\\
		\> \emph{Description}
  \end{tabbing}
\end{quote}
where the \emph{Description} should be as detailed as possible.
After the first error,
your compiler may either make a ``best effort''
attempt to continue processing the input file, or exit.
Your compiler should continue to catch and report errors
at the lexing and parsing stages
(i.e., errors from Parts 1 and \parser{} should remain).

If there are no errors,
then the output should be a report
listing each global variable
and function definition,
and for each function,
listing its return type,
list of parameters,
local variables,
for each statement of the form
\begin{quote}
\emph{expression} ;
\end{quote}
indicating the type of the expression.
See Section~\ref{SEC:output}
for more details.
If the input file contains errors,
it is ok to generate a partial report
before the error is discovered.

%======================================================================
\section{Type checking} \label{SEC:types}
%======================================================================


\subsection{Literals, identifiers, and functions}

Each expression will have a corresponding type.
For literals, the type may be inferred based on the literal:
character literals have type {\tt char},
integer literals have type {\tt int},
real literals have type {\tt float},
and string literals have type {\tt char[]}.
These may be modified by {\tt const}, see Section~\ref{SEC:constants}.
For global variables, local variables, and parameters,
the type should match how they are declared.
The return type of a function specifies the type of any call to that
function.
Regarding functions and identifiers,
an appropriate error message should be given for
\begin{itemize}
  \item
  using a variable that has not been declared;

  \item
  declaring a local variable with the same name as another
  local variable or parameter of the same function;

  \item
  declaring a parameter with the same name as another parameter
  in the same prototype;

  \item
  declaring a global variable with the same name as another
  global variable;

  \item
  calling a function that has not been defined or declared as a prototype;

  \item
  calling a function with the incorrect number of parameters;

  \item
  calling a function with passed parameters whose types
  do not \emph{exactly} match the parameters given in its prototype
    (see Section~\ref{SEC:widening} to relax this slightly);

  \item
  declaring two prototypes for the same function name
  that are not exactly the same (ignoring parameter names);

  \item
  a {\tt return} statement inside a function,
  that returns a value with a different type than the function's type
  (do not worry yet about \emph{missing} {\tt return} statements);

  \item
  giving more than one definition for the same function name.

\end{itemize}
Note that these type checking rules are more strict than standard C.
Extra credit options in Sections~\ref{SEC:widening} and following
may adjust these rules or (more likely)
introduce additional rules,
to bring type checking closer to the C standard.


\subsection{Expressions}

\begin{table}[t]
\centering
\begin{tabular}{cccc|c}
  \# &
  \multicolumn{3}{c|}{ Operation } & Result type
\\ \hline
  \rulenumber{ite} &
  \multicolumn{3}{c|}{ $N$ ~ {\tt ?} ~ $T$ ~ {\tt :} ~ $T$ } & $T$
\\
  \multicolumn{4}{c}{~}
\\[2mm]
  \# &
  Left type & operator & Right type & Result type
\\ \hline
  \rulenumber{tilde}  &       & \verb|~| & $I$ & $I$ \\[1mm]
  \rulenumber{negate} &       & {\tt -} & $N$ & $N$ \\[1mm]
  \rulenumber{not}    &       & {\tt !} & $N$ & {\tt char}  \\[1mm]
  \rulenumber{makec}  &       & {\tt (char)} & $N$ & {\tt char} \\[1mm]
  \rulenumber{makei}  &       & {\tt (int)}  & $N$ & {\tt int}  \\[1mm]
  \rulenumber{makef}  &       & {\tt (float)} & $N$& {\tt float}
    \\[2mm]
  \rulenumber{mod} & $I$ &
    {\tt \%}, {\tt \&}, {\tt |}
    & $I$ & $I$
  \\[1mm]
  \rulenumber{plus} & $N$ &
    {\tt +}, {\tt -}, {\tt *}, {\tt /}
    & $N$ & $N$
  \\[1mm]
  \rulenumber{compare} & $N$ &
    {\tt ==}, {\tt !=}, {\tt >}, {\tt >=}, {\tt <}, {\tt <=},
    {\tt \&\&}, {\tt ||}
    & $N$ & {\tt char}
  \\[2mm]
  \rulenumber{preinc} &  &
    {\tt ++}, {\tt --}
    & $N$ & $N$
  \\[1mm]
  \rulenumber{postinc} & $N$ &
    {\tt ++}, {\tt --}
    & & $N$
  \\[1mm]
  \rulenumber{assign} & $N$ &
    {\tt =}, {\tt +=}, {\tt -=}, {\tt *=}, {\tt /=}
    & $N$ & $N$
  \\[2mm]
  \rulenumber{array} & $T${\tt[]} &
    {\tt [ int ]}
    & & $T$
\end{tabular}
\caption{Operations on types.
  $I \in \{\mathtt{char}, \mathtt{int}\}$ is any integer type,
  $N \in \{\mathtt{char}, \mathtt{int}, \mathtt{float}\}$ is any numeric type,
  and
  $T$ is \emph{any} type.
} \label{TAB:types}
\end{table}

All expressions should be checked for type errors.
Table~\ref{TAB:types} shows
how operators are defined for various types of operands,
and the resulting type.
Rules are numbered for later reference.
It also shows which explicit coercions, or casts, are allowed
(c.f. rules R\ref{RULE:makec}, R\ref{RULE:makei}, and R\ref{RULE:makef}).
Note that the variables $I$, $N$, and $T$ represent the same type
within a rule;
for example,
  rule~R\ref{RULE:plus} says the result of
  {\tt char + char} has type {\tt char}.
For a minimal implementation,
you may give an error if the operand types
do not \emph{exactly} match.
For example,
  rule~R\ref{RULE:plus} says the operands of {\tt +} must be
  the same type;
  thus an operation {\tt int + char} would produce a type error.
Rule~R\ref{RULE:array} is for array indexing,
and states both that
  it is an error to index an item that is not an array type,
  and that
  array indices must be integers (specifically, {\tt int}s).

%
% Save for code generation
%
% \subsection{Statements}
% \label{SEC:stmts}

% Statements that require a condition
% must check that the expression given for the condition
% is a numerical type (one of {\tt char}, {\tt int}, or {\tt float}).
% Specifically, this means to check the following.
% \begin{itemize}
  % \item The expression inside an {\tt if} condition is numerical.
  % \item The expression inside a {\tt while} or {\tt do}--{\tt while}
        % condition is numerical.
  % \item The second expression in a {\tt for} loop
        % (that determines if the loop continues or not),
        % \emph{if present}, must be numerical.
% \end{itemize}

\subsection{Extra credit: widening}
\label{SEC:widening}

\begin{table}[t]
\centering
  \begin{tabular}{c|c}
    Original & Widened \\ \hline
    {\tt char} & {\tt int} \\
    {\tt int}  & {\tt float}
  \end{tabular}
\caption{Allowed type widenings.} \label{TAB:widenings}
\end{table}

Relax the type checking rules slightly by
automatically coercing types,
by widening only (according to Table~\ref{TAB:widenings})
as necessary.
Note that more than one widening may be possible, and desired.
For example, for operation {\tt char + float},
the left operand of type {\tt char} can be widened to {\tt int}
which can then be widened to {\tt float}.
Once the left operand has automatically been widened to type {\tt float},
rule~R\ref{RULE:plus}
  indicates that the resulting type is {\tt float}.

For extra credit,
your compiler should automatically widen types any place it is needed:
for operations shown in Table~\ref{TAB:types},
for assignments,
for parameter passing,
and for function return values.
For example, if we have the following function defined:
\begin{quote}
  \begin{lstlisting}
    float widen(int a)
    {
      a += 'x';
      return a;
    }
  \end{lstlisting}
\end{quote}
then the expression in line~3 would automatically widen
the character constant \verb|'x'| to type {\tt int},
and the return statement in line~4 would automatically
widen variable {\tt a} from type {\tt int} to type {\tt float}
as required for the return type of function {\tt widen()}.
Calling the function as \verb|widen('q')|
would be allowed, as the character \verb|'q'| can
be automatically widened to type {\tt int} as required.


\subsection{Extra credit: initialization}
\label{SEC:initialize}

Implement type checking for variable initialization.
Note that this requires your parser to support variable initialization,
which was extra credit for part~\parser.
Note that several variables may be declared and initialized at once:
\begin{quote}
  \begin{lstlisting}[numbers=none]
    void foo(int a)
    {
      int b = a,      /* OK */
          c,
          d = 0,      /* OK */
          e = d,      /* OK */
          f = 3.14,   /* ERROR, type mismatch */
          g;
    }
  \end{lstlisting}
\end{quote}

\subsection{Extra credit: constants}
\label{SEC:constants}

Implement type checking for the {\tt const} modifier.
Note that this requires your parser to support the {\tt const}
modifier, which was extra credit for part~\parser.

For this extra credit,
an appropriate error message should be generated
if any {\tt const} item is used as an ``l-value'';
i.e., the left side of any assignment opperator
in rule~R\ref{RULE:assign},
and any item with an increment or decrement operator
(rules R\ref{RULE:preinc} and R\ref{RULE:postinc})
has the {\tt const} property.
The only exception is for variable initializations during declaration
(if this has been impemented):
\begin{quote}
  \begin{lstlisting}[numbers=none]
    void bar(int a)
    {
      const int b = a;  /* OK, declare and initialize  */
      const int c;
      c = 4;            /* ERROR, assignment to a const */
    }
  \end{lstlisting}
\end{quote}
Additionally,
character, integer, real, and string literals
should have their ordinary types but modified with {\tt const},
and rules R1 through R10 in Table~\ref{TAB:types}
should be updated so that, if all operands are {\tt const},
then the result type should also be {\tt const}.
For example, the expressions \verb|17|, \verb|3+(4*5)|,
and \verb|(1>2)?3:4|
should all have type {\tt const int}.


\subsection{Extra credit: user-defined structs}
\label{SEC:userstructs}

Implement type checking for user-defined structs.
Note that this requires your parser to support user-defined structs,
which was extra credit for part~\parser.
This means that appropriate error messages should be given for the following.
\begin{itemize}
  \item
  It is an error to define two global structs with the same name.

  \item
  It is an error to define two local structs with the same name.

  \item
  When declaring a global variable, local variable,
  or parameter that is a struct type,
  it is an error if that struct type has not been defined.
\end{itemize}



\subsection{Extra credit: struct member selection}
\label{SEC:members}

Implement type checking for member selection of structs.
Note that this requires your parser to support member selection
with the `` {\tt .} '' operator,
which was extra credit for part~\parser.
Practically, it also requires you to implement user-defined structs
as discussed in Section~\ref{SEC:userstructs},
otherwise this will be impossible to test.

This means to implement the following type checking rule
\begin{center}
  \begin{tabular}{cc|c}
    \# & ~ \qquad Operation \qquad ~ & Result type
  \\ \hline
    \rulenumber{members}
    & $S$ ~\verb|.|~ m & $T$
  \end{tabular}
\end{center}
where $T$ is the type of member ``m'' in struct $S$.
That is,
  for any expression of the form ``item.m'':
\begin{itemize}
  \item
  It is a type error if ``item'' is not a {\tt struct} type.

  \item
  It is a type error if the struct ``item''
  does not have a member named ``m''.

  \item
  Otherwise, ``item.m''
  has type as given by the type of member ``m'',
  as defined in the struct corresponding to ``item''.
\end{itemize}
Note that this is complicated by the fact that ``item''
may also involve member selection and array indexing.
This portion of extra credit is to check expressions of the form
\begin{quote}
\begin{lstlisting}[numbers=none]
  window[3].upperleft.x = mouse.x.stack[mouse.x.top];
\end{lstlisting}
\end{quote}


%======================================================================
\section{Format of output} \label{SEC:output}
%======================================================================

Output should be a report for global variable declarations,
global structure declarations, and function definitions
({\bf not} prototypes),
in the same order they appear in the input file.
Students are encouraged to indent lines to make the output easier to read,
and to check the example outputs
in Sections \ref{SEC:exmin} and \ref{SEC:exmax}.

\subsection{Report for each variable}

For each defined variable,
write a line of the form
\begin{quote}
  {\tt Line } \emph{linenum}{\tt :}~\emph{defined}~\emph{type}~\emph{ident}\verb|[|\emph{dim}\verb|]|
\end{quote}
to the output stream,
where \emph{linenum} is the line number in the input file where the
declaration appears (based on the location of the identifier token);
\emph{defined} is one of ``\verb|local|'' for local variables,
  ``\verb|global|'' for global variables,
  ``\verb|member|'' for struct members (see Section~\ref{SEC:structout} below),
  or
  ``\verb|parameter|'' for function parameters (see Section~\ref{SEC:funcout} below);
\emph{type} is the type of the variable;
\emph{ident} is the variable name;
and the array dimension is given only for array variables
(the dimension should be empty for array parameters).
For example,
the source lines
\begin{quote}
  \begin{lstlisting}[firstnumber=2]
  /* Global variables */
  int   x1, y1, /* upper-left corner */
        x2, y2, /* lower-right corner */
        A[50],
        i, j, k;
  \end{lstlisting}
\end{quote}
should generate the output:
\begin{quote}
  \begin{lstlisting}[style=Output]
  Line  3: global int x1
  Line  3: global int y1
  Line  4: global int x2
  Line  4: global int y2
  Line  5: global int A[50]
  Line  6: global int i
  Line  6: global int j
  Line  6: global int k
  \end{lstlisting}
\end{quote}


\subsection{Report for a structure declaration}
\label{SEC:structout}

For each struct definition
(assuming extra credit described in Section~\ref{SEC:userstructs}
has been implemented),
write a line of the form
\begin{quote}
  {\tt Line } \emph{linenum}{\tt :}~\emph{defined}~{\tt struct}~\emph{ident}
\end{quote}
to the output stream,
where \emph{linenum} is the line number in the input file where
the struct is defined
(based on the location of the struct name),
\emph{defined} is one of ``\verb|local|'' for local structs,
  or ``\verb|global|'' for global structs,
and \emph{ident} is the struct name.
Then, generate a line for each struct member.
For example, the source lines
\begin{quote}
\begin{lstlisting}[firstnumber=8]
  struct message {
      int x, y;       /* message coordinates */
      char text[256]; /* message contents */
  };
\end{lstlisting}
\end{quote}
should generate the output:
\begin{quote}
  \begin{lstlisting}[style=Output]
  Line  8: global struct message
      Line  9: member int x
      Line  9: member int y
      Line 10: member char text[256]
  \end{lstlisting}
\end{quote}

\subsection{Report for an expression statement}
\label{SEC:exprout}

For each statement of the form
\begin{quote}
  \emph{expression} {\tt ;}
\end{quote}
write a line of the form
\begin{quote}
  {\tt Line } \emph{linenum}{\tt : expression has type}~\emph{type}
\end{quote}
to the output stream,
where \emph{linenum} is the line number of the expression statement
(based on the location of the ``\verb|;|'')
and \emph{type} is the type of the expression,
as one of the four basic types
{\tt void}, {\tt char}, {\tt int}, or {\tt float},
or a user-defined {\tt struct structname}
(if extra credit has been implemented),
followed by ``\verb|[]|'' if it is an array.
For example, the source lines
\begin{quote}
\begin{lstlisting}[firstnumber=23]
    putchar(72);
    3.14159265 / 2.0;
    "Hello, world!";
\end{lstlisting}
\end{quote}
should generate the output:
\begin{quote}
  \begin{lstlisting}[style=Output]
  Line 23: expression has type int
  Line 24: expression has type float
  Line 25: expression has type char[]
  \end{lstlisting}
\end{quote}




\subsection{Report for a function definition}
\label{SEC:funcout}

For each function definition (prototype plus function body),
write a line of the form
\begin{quote}
  {\tt Line } \emph{linenum}{\tt :}~{\tt function}~\emph{type}~\emph{ident}
\end{quote}
to the output stream,
where \emph{linenum} is the line number where the function is defined
(based on the location of the function name),
\emph{type} is the return type of the function,
and \emph{ident} is the function name.
Then, generate a line for each function parameter,
local variable, local struct,
and expression statement
within the function body,
in the order in which they appear in the source file.
For example, the source lines
\begin{quote}
  \begin{lstlisting}[firstnumber=55]
    int sumelems(int A[], int n)
    {
      int i, sum;
      sum = 0;
      for (i=0; i<n; ++i) {
        sum += A[i];
      }
      return sum;
    }
  \end{lstlisting}
\end{quote}
should generate the output:
\begin{quote}
  \begin{lstlisting}[style=Output]
    Line 55: function int sumelems
        Line 55: parameter int A[]
        Line 55: parameter int n
        Line 57: local int i
        Line 57: local int sum
        Line 58: expression has type int
        Line 60: expression has type int
  \end{lstlisting}
\end{quote}


%======================================================================
\section{Examples: minimal implementation} \label{SEC:exmin}
%======================================================================

\subsection{Input: {\tt test1.c}}

\lstinputlisting{test1.c}

\subsection{Output for {\tt mycc -\typecheck{} test1.c}}

\lstinputlisting[style=Output]{test1.c.output4}

\subsection{Input: {\tt test2.c}}

\lstinputlisting{test2.c}

\subsection{Errors for {\tt mycc -\typecheck{} test2.c}}

\lstinputlisting[style=Output]{test2.c.error4}

\subsection{Input: {\tt test3.c}}

\lstinputlisting{test3.c}

\subsection{Errors for {\tt mycc -\typecheck{} test3.c}}

\lstinputlisting[style=Output]{test3.c.error4}

\subsection{Input: {\tt test4.c}}

\lstinputlisting{test4.c}

\subsection{Errors for {\tt mycc -\typecheck{} test4.c}}

\lstinputlisting[style=Output]{test4.c.error4}

%======================================================================
\section{Extra credit examples} \label{SEC:exmax}
%======================================================================

\subsection{Input: {\tt extra1.c} (widening)}

\lstinputlisting{extra1.c}

\subsection{Errors for bare implementation}

\lstinputlisting[style=Output]{extra1.c.error4}

\subsection{Output for extra credit implementation}

\lstinputlisting[style=Output]{extra1.c.extra.output4}

\subsection{Input: {\tt extra2.c}}

\lstinputlisting{extra2.c}

\subsection{Output for extra credit implementation}

\lstinputlisting[style=Output]{extra2.c.extra.output4}

\subsection{Input: {\tt extra3.c}}

\lstinputlisting{extra3.c}

\subsection{Errors for extra credit implementation}

\lstinputlisting[style=Output]{extra3.c.extra.error4}

\subsection{Input: {\tt extra4.c}}

\lstinputlisting{extra4.c}

\subsection{Output for extra credit implementation}

\lstinputlisting[style=Output]{extra4.c.extra.output4}

\subsection{Input: {\tt extra5.c}}

\lstinputlisting{extra5.c}

\subsection{Errors for extra credit implementation}

\lstinputlisting[style=Output]{extra5.c.extra.error4}



%======================================================================
\section{Grading}
%======================================================================

\noindent
\begin{gradetable}
  \mainitem{16}{Documentation}
  \\[1mm]
  \inneritem{4}{\tt README.txt}
  \\[1mm]
  \innerpara{%
    How to build the compiler and documentation.
    Updated to show which part 3 features are implemented.
  }
  \\[1mm]
  \inneritem{12}{\tt developers.pdf}
  \\[1mm]
  \innerpara{%
    New section for part \typecheck, that explains
    the purpose of each source file,
    the main data structures used (or how they were updated),
    and gives a high-level overview of how the various
    features are implemented.
  }
  \\[4mm]

  \mainitem{8}{Ease of grading}
  \\[1mm]
  \inneritem{4}{Building}
  \\[1mm]
  \innerpara{%
    How easy was it for the graders to build your compiler and
    documentation?
    For full credit, simply running ``{\tt make}''
    should build both the documentation and the compiler executable,
    and running ``{\tt make clean}'' should remove
    all generated files.
  }
  \\[1mm]
  \inneritem{4}{Works with script}
  \\[1mm]
  \innerpara{%
    Does the {\tt -o} switch work?
    Is your output formatted correctly?
  }
  \\[4mm]

  \mainitem{10}{Still works in modes 0, 1, and \parser}
  \\[4mm]

  \mainitem{66}{Type checking}
  \\[1mm]
  \inneritem{4}{Literals}
  \\[1mm]
  \inneritem{9}{Identifiers (global variables, local variables, parameters)}
  \\[1mm]
  \inneritem{9}{Function calls}
  \\[1mm]
  \inneritem{4}{Function returns}
  \\[1mm]
  \inneritem{4}{Detects duplicate or mismatched function prototypes}
  \\[1mm]
  \inneritem{4}{Unary operators}
  \\[1mm]
  \inneritem{4}{Casts}
  \\[1mm]
  \inneritem{6}{Binary operators: arithmetic}
  \\[1mm]
  \inneritem{6}{Binary operators: comparison and logic}
  \\[1mm]
  \inneritem{4}{Assignment and update operators}
  \\[1mm]
  \inneritem{4}{Increment and decrement}
  \\[1mm]
  \inneritem{4}{Array indexing}
  \\[1mm]
  \inneritem{4}{Ternary operator}
  \\[1mm]

  \mainitem{5}{Extra credit: widening}
  \\[1mm]
  \mainpara{%
    See Section~\ref{SEC:widening}.
  }
  \\[4mm]

  \mainitem{5}{Extra credit: initializations}
  \\[1mm]
  \mainpara{%
    See Section~\ref{SEC:initialize}.
  }
  \\[4mm]
  \mainitem{5}{Extra credit: constants}
  \\[1mm]
  \mainpara{%
    See Section~\ref{SEC:constants}.
  }
  \\[4mm]

  \mainitem{10}{Extra credit: user-defined structs}
  \\[1mm]
  \mainpara{%
    See Section~\ref{SEC:userstructs}
  }
  \\[4mm]

  \mainitem{5}{Extra credit: struct member selection}
  \\[1mm]
  \mainpara{%
    See Section~\ref{SEC:members}.
  }
  \\[4mm]

  \gradeline
  \mainitem{100}{Total for students in 440 (max points is 120)}
  \\
  \mainitem{115}{Total for students in 540}
\end{gradetable}


\section{Submission}

\begin{table}[h]
\centering

  \begin{tabular}{lcr}
    {\bf Part} & ~~~~ & {\bf Penalty applied} \\ \cline{1-1}\cline{3-3}
    Part 0 && 30\% off \\
    Part 1 && 20\% off \\
    Part \parser && 10\% off
  \end{tabular}

\caption{Penalty applied when re-grading}
\label{TAB:penalties}
\end{table}

Be sure to commit your source code and documentation to your
git repository, and to upload (push) those commits to the server
so that we may grade them.
In Canvas,
indicate which parts you would like us to re-grade for reduced credit
(see Table~\ref{TAB:penalties} for penalty information).
Otherwise, we will grade only part \typecheck.




\end{document}

