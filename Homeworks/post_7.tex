\documentclass[10pt]{article}
\usepackage{fullpage}
\usepackage{url}
\usepackage{color}
\usepackage{listings}

\definecolor{dkgreen}{rgb}{0,0.6,0}
\definecolor{dkred}{rgb}{0.6,0,0}
\definecolor{dkblue}{rgb}{0,0,0.7}

\usepackage{tikz}
\usetikzlibrary{automata, positioning, arrows}
\tikzset{%
  node distance=2.5cm,  
  initial text={},      
  every state/.style={  
    semithick},
  double distance=2pt,  % Accept state appearance
  every edge/.style={   
    draw,
    ->,
    >=stealth',
    auto,
    semithick} %
}

\lstdefinestyle{jvm}{
  % aboveskip=3mm,
  % belowskip=3mm,
  xleftmargin=2em,
  % xrightmargin=2em,
  showstringspaces=false,
  columns=flexible,
  basicstyle={\ttfamily},
  numbers=left,
  moredelim=[s][\color{black}]{Ljava}{;},
  morecomment=[l][\color{dkgreen}]{;},
  morecomment=[l][\color{magenta}]{;;},
  keywords={class,public,static,super,method,code,end},
  keywordstyle=\color{dkblue},
  breaklines=true,
  breakatwhitespace=true,
  tabsize=8
}

\renewcommand{\thepage}{~}

\title{COM S 440/540 Homework 7}
\date{}
\author{Code generation}

\begin{document}

\maketitle

\noindent
Reminder: present your own work and properly cite any sources used.
Solutions should be written satisfying the \emph{other student viewpoint},
and must be prepared using \LaTeX.
\renewcommand{\thepage}{~}
%============================================================
\section*{Question~1~\hfill 15--20 points}
%============================================================

Suppose we define a new unary operator, \verb|square()|,
that squares its operand.
More formally, if we have an expression of the form
\[
E ~\rightarrow~~\verb|square|~\verb|(|~ E_1 ~\verb|)|
\]
it means to evaluate the expression $E_1$ exactly once,
and compute the value of the expression multiplied by itself.
Give the semantic rules for this production,
to (incrementally) translate into three-address code.
For additional points, also give the semantic rules
to (incrementally) translate into stack machine code.

%============================================================
\section*{Question~2~\hfill 20--25 points}
%============================================================

Suppose we are designing a language with short-circuiting
boolean expressions,
and wish to include an \emph{implication} operator, \verb|=>|,
with the following truth table:
\[
  \begin{array}{c|c|c}
    x & y & x ~\verb|=>|~ y
  \\ \hline
    \mathit{false} & \mathit{false} & \mathit{true}
  \\
    \mathit{false} & \mathit{true} & \mathit{true}
  \\
    \mathit{true} & \mathit{false} & \mathit{false}
  \\
    \mathit{true} & \mathit{true} & \mathit{true}
  \end{array}
\]
Give semantic rules
for the production
\[
  B ~\rightarrow~ B_1 ~~\verb|=>|~~ B_2
\]
with short-circuiting.
For additional points,
give the semantic rules using the special fall-through label,
to avoid redundant gotos.

%============================================================
\section*{Question~3~\hfill 20--30 points}
%============================================================
Suppose a \emph{repeat-until} loop
has the form
\[
  S \quad \rightarrow \quad \mathsf{repeat} ~~ S_1 ~~\mathsf{until} ~~ B
\]
where $B$ is a boolean expression that short circuits,
and $S_1$ is a non-empty list of statements.
The loop will execute statement(s) $S_1$ one or more times,
stopping when condition $B$ is true.
Give the semantic rules to generate three-address code for a repeat-until loop.
For additional points, also give the semantic rules using backpatching
(you may modify the above production to include markers if necessary).

\end{document}
