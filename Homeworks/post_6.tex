\documentclass[10pt]{article}
\usepackage{fullpage}
\usepackage{url}
\usepackage{color}
\usepackage{listings}

\definecolor{dkgreen}{rgb}{0,0.6,0}
\definecolor{dkred}{rgb}{0.6,0,0}
\definecolor{dkblue}{rgb}{0,0,0.7}

\usepackage{tikz}
\usetikzlibrary{automata, positioning, arrows}
\tikzset{%
  node distance=2.5cm,  
  initial text={},      
  every state/.style={  
    semithick},
  double distance=2pt,  % Accept state appearance
  every edge/.style={   
    draw,
    ->,
    >=stealth',
    auto,
    semithick} %
}

\lstdefinestyle{jvm}{
  % aboveskip=3mm,
  % belowskip=3mm,
  xleftmargin=2em,
  % xrightmargin=2em,
  showstringspaces=false,
  columns=flexible,
  basicstyle={\ttfamily},
  numbers=left,
  moredelim=[s][\color{black}]{Ljava}{;},
  morecomment=[l][\color{dkgreen}]{;},
  morecomment=[l][\color{magenta}]{;;},
  keywords={class,public,static,super,method,code,end},
  keywordstyle=\color{dkblue},
  breaklines=true,
  breakatwhitespace=true,
  tabsize=8
}

\renewcommand{\thepage}{~}

\title{COM S 440/540 Homework 6}
\date{}
\author{Attribute grammars}

\begin{document}

\maketitle

\noindent
Reminder: present your own work and properly cite any sources used.
Solutions should be written satisfying the \emph{other student viewpoint},
and must be prepared using \LaTeX.
\renewcommand{\thepage}{~}
%============================================================
\section*{Question~1~\hfill 20 points}
%============================================================

\begin{eqnarray}
  S & \rightarrow & S ~+~ F
\\
  S & \rightarrow & F
\\
  F & \rightarrow & \mathit{number}
\\
  F & \rightarrow & \mathit{ident} ~(~ S ~)
\end{eqnarray}
For the grammar above, give semantic rules to build
a synthesized attribute named \emph{max}, 
which keeps track of the largest \emph{number} appearing in the expression.
You may introduce other attributes,
and give rules for them,
as needed.
For example, the sentence
\[
1 + f(~g(5+4) + h(2) + 3~)
\]
should have a \emph{max} attribute equal to 5.

%============================================================
\section*{Question~2~\hfill 30 points}
%============================================================

\begin{eqnarray}
  \mathit{func} & \rightarrow & \mathit{header} ~ \mathit{block}
\\
  \mathit{block} & \rightarrow & \{ ~ \mathit{slist} ~ \}
\\
  \mathit{slist} & \rightarrow & \mathit{slist} ~ \mathit{stmt}
\\
  \mathit{slist} & \rightarrow & \epsilon
\\
  \mathit{stmt} & \rightarrow & \mathit{expr} ~;
\\
  \mathit{stmt} & \rightarrow & \mathbf{while} ~ ( ~ \mathit{expr} ~ ) ~ \mathit{block}
\end{eqnarray}
For the grammar above, give semantic rules to build 
an inherited attribute named \emph{depth},
which keeps track of the depth of each statement
(grammar symbol \emph{stmt})
in terms of braces.
For example, for the input
\begin{verbatim}
  int main()
  {
    x=0;
    while(x<10) {
      i=x;
      while (i-- >= 0) { putchar(' '); }
      printf("%d\n", x++);
    }
  }
\end{verbatim}
statement ``\verb|while(x<10)|'' has depth~1,
statement ``\verb|i=x|'' has depth~2, 
and statement ``\verb|putchar(' ');|'' has depth~3.


\end{document}
