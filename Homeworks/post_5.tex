\documentclass[10pt]{article}
\usepackage{fullpage}
\usepackage{url}
\usepackage{color}
\usepackage{listings}

\definecolor{dkgreen}{rgb}{0,0.6,0}
\definecolor{dkred}{rgb}{0.6,0,0}
\definecolor{dkblue}{rgb}{0,0,0.7}

\usepackage{tikz}
\usetikzlibrary{automata, positioning, arrows}
\tikzset{%
  node distance=2.5cm,  
  initial text={},      
  every state/.style={  
    semithick},
  double distance=2pt,  % Accept state appearance
  every edge/.style={   
    draw,
    ->,
    >=stealth',
    auto,
    semithick} %
}

\lstdefinestyle{jvm}{
  % aboveskip=3mm,
  % belowskip=3mm,
  xleftmargin=2em,
  % xrightmargin=2em,
  showstringspaces=false,
  columns=flexible,
  basicstyle={\ttfamily},
  numbers=left,
  moredelim=[s][\color{black}]{Ljava}{;},
  morecomment=[l][\color{dkgreen}]{;},
  morecomment=[l][\color{magenta}]{;;},
  keywords={class,public,static,super,method,code,end},
  keywordstyle=\color{dkblue},
  breaklines=true,
  breakatwhitespace=true,
  tabsize=8
}

\renewcommand{\thepage}{~}

\title{COM S 440/540 Homework 5}
\date{}
\author{LR Parsing}

\begin{document}

\maketitle

\noindent
Reminder: present your own work and properly cite any sources used.
Solutions should be written satisfying the \emph{other student viewpoint},
and must be prepared using \LaTeX.
\renewcommand{\thepage}{~}
%============================================================

\begin{eqnarray}
  S & \rightarrow & S ~+~ F
\\
  S & \rightarrow & F
\\
  F & \rightarrow & \mathit{number}
\\
  F & \rightarrow & \mathit{ident} ~(~ S ~)
\end{eqnarray}


%============================================================
\section*{Question~1~\hfill 10 points}
%============================================================

Determine the LR(0) \textsc{Closure} of item ~~
$F$ $\rightarrow$ $\mathit{ident}$ $($ $\bullet$ $S$ $)$

%============================================================
\section*{Question~2~\hfill 10 points}
%============================================================

Using {\tt bison -g}, generate a DOT file of the automaton
for this grammar.
Then, use {\tt dot -Tpdf} to produce a PDF of the automaton.
Either include this in your \LaTeX{} submission, or submit it separately
in Canvas.


%============================================================
\section*{Question~3~\hfill 45 points}
%============================================================

Using the automaton states as numbered by {\tt bison},
give the LR(0) \textsc{Action} and \textsc{Goto} tables
for this grammar.
Indicate if there are any conflicts.


%============================================================
\section*{Question~4~~(extra credit for students in 440)\hfill 10 points}
%============================================================

Trace the execution of your LR(0) parser on the input string
\[
\mathit{number} ~+~ \mathit{number} ~)~ \mathit{ident} ~\$
\]
by indicating the stack contents, remaining input, and
action(s) taken at each step.

\end{document}
